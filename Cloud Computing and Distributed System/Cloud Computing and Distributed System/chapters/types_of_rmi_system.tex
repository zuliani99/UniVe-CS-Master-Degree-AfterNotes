\chapter{Types of RMI system}
\begin{itemize}
    \item \textbf{Common Object Broker Architecture (CORBA):}
        \begin{itemize}
            \item Allows the communication among applications written in \textbf{various languages} through the use of a model of neutral object
            \item Allows the communication among existing applications developed in \textbf{heterogeneous environments} and for which one cannot make a conversion in a common language
        \end{itemize}
    \item \textbf{Java RMI:}
        \begin{itemize}
            \item Allows the communication among \textbf{applications written in Java}
            \item \textbf{Environment heterogeneity} as a consequence of the Java language portability. It leads to a simpler code 
        \end{itemize}
\end{itemize}

\section{Distributed object model features}
\begin{itemize}
    \item The clients of remote objects interact through remote interface and must manage the exception for possible failure of RMI 
    \item The arguments and the results of the remote methods are passed by value and not by reference, by using the concept of serialisation
    \item A remote object is always passed by reference
    \item There are some security mechanisms introduced to check the behaviour of the classes and the references
\end{itemize}


\section{Garbage collection of remote objects}
\begin{itemize}
    \item Every JVM updates a series of counters, each of them associated to a given object
    \item Each counter represents the reference number to a given object that is currently active on a JVM
    \item Every time a reference to a remote object is created, the corresponding counter is incremented
    \item When no client has reference to an object, the runtime of RMI uses a “weak reference” to address it 
    \item The weak reference is used by the garbage collector to cancel the object as soon as also the local references are not present
\end{itemize}

\section{Serialisation}
\begin{itemize}
    \item Mechanism used for the data transmission between client and server in RMI 
    \item It is an automatic transformation of objects and structures of objects in simple byte sequences, to be sent in a data stream
    \item There can be serialisation only for instances of serializable object
    \item It is not transferred the real object, but only the information that characterise the instance of it
    \item At de-serialization time a copy will be created of the transmitted instance by using the .class
\end{itemize}