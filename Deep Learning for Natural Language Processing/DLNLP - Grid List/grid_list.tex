\chapter{Langauge Model}
\begin{itemize}
    \item \textbf{Probabilistic Language Modeling}
    \begin{itemize}
        \item \textbf{Goal}: compute the probability of a sentence or sequence of words 
        \item \textbf{Related task}: probability of upcoming words
        \item \textbf{A model}: that compute either of these \(p(w)\) or \(p(w_n | w_1, w_2, ..., w_{n-1})\) is a LM or language model
        \item How to compute \(p(w)\)? \(\rightarrow\) chin rule of probability
        \item How to estimates these probabilities?
        \begin{itemize}
            \item count / count minus the specific word
            \item \textit{Any issues?} 
            \begin{itemize}
                \item Word that cannot appear
                \item Very expensive
                \item Infinite number of prefix so a very sparse prob
            \end{itemize}
        \end{itemize}
    \end{itemize}
    \item Unigram Model
    \item Text generation with a LM
    \begin{itemize}
        \item Given the prefix how to pick the next given word? Randomly, sampling word from the distribution, pick word with higher conditional probability
        \item Once I choose the word there is no way of making an correction
    \end{itemize}
    \item N-grams model
    \begin{itemize}
        \item We can extend to trigram, 4-grams, 5-grams
        \item insufficient model of language since it has long dependencies
    \end{itemize}
    \item 2-gram model 
    \begin{itemize}
        \item \[p(w_i, w_{i-1}) = \frac{count(w_{i-1}, w_i)}{count(w_{i-1})}\]
    \end{itemize}
    \item Estimating probabilities
    \begin{itemize}
        \item Instead of doing multiplication we can use a summation by the logarithm
    \end{itemize}
    \item \textbf{How good is our LM}
    \begin{itemize}
        \item Consider the perplexity: the best ML is the one that best predict an unseen test set
        \item Perplexity is the inverse probability of the test set, normalized by the number of words
        \item Minimizing the perplexitu is the same as maximising the probability 
        \item Lower perplexity = better model
        \item Bigram with zero probability: 0 prob to the test set so we cant compute the perplexity
        \item How we deal with n grams of zero probabilities?
        \begin{itemize}
            \item Add one estimation, called laplace smoothing, consist on pretend that we saw each word one more time that we did (add one to the count)
            \item One hot encoding, but dot product is zero, vectors are orthogonal
        \end{itemize}
    \end{itemize}
\end{itemize}



\chapter{NL Model}
\begin{itemize}
    \item Represent words with low dimensional vectors called embeddings
    \item Given the embedding through a NN, softmax layer convert a vector into a probability distribution over the entire vocabulary
    \item \textbf{Fixed windows approach}
    \begin{itemize}
        \item words OHE
        \item concatenated word embeddings
        \item hidden layer
        \item output soft max
        \begin{itemize}
            \item Advantages:
            \begin{itemize}
                \item No sparsity problem
                \item Model is size O(n)
            \end{itemize}
            \item Disadvantages:
            \begin{itemize}
                \item Fixed window is too small
                \item Enlarging window enlarged W
                \item Location of word matters we wold like something more general
            \end{itemize}
        \end{itemize}
    \end{itemize}
    \item Relations
    \begin{itemize}
        \item Synonyms
        \item Similarity
        \item Relatedness
        \item Anatomy
        \item Connotation
    \end{itemize}
    \item \textbf{Word meaning}
    \begin{itemize}
        \item Defining a meaning by linguistic distribution
        \item Each word is a vector called embedding
        \item Vectors since
        \begin{itemize}
            \item With a word, a feature is a word identity
            \item With \textbf{embeddings}, feature is a word vector, thus we can generalize similar but unseen words
        \end{itemize}
    \end{itemize}
    \item \textbf{OHE:} 
    \begin{itemize}
        \item Sentence are represented as a 0/1 vector of dimension the the length of the vocabulary
        \item Similarity via dot product not work since vector are orthogonal
        \item RMSE works fine
    \end{itemize}
    \item Distributional encoding (bag of words): words that occur in similar context has similar meaning
    \item TF-IDF
    \item \textbf{Similarity measures}:
    \begin{itemize}
        \item Dot product: tends to be higher when the two vectors have large values in the same dimension, so we normalize it by the length
        \item Cosine similarity
    \end{itemize}
    \item \textbf{CBOW}: given a window of words of length \(2m + 1\) define a probabilistic model for predicting the middle word
    \[P(x_0 | x_{-m},..., x_{-1}, x_1, ..., x_m)\]
    Train the model minimizing the loss
    \[L = - \sum \log P(x_{-m},..., x_{-1}, x_1, ..., x_m)\]
    \begin{itemize}
        \item Map all the context words into the n dimensional space
        \item Average these vectors to get a context vector
        \item Get the score vector from the output
        \item Use the score to compute probability via softmax
    \end{itemize}
    \item \textbf{Skip-Gram:} given a window of words of length \(2m + 1\) define a probabilistic model for predicting each context word
    \[P(x_{context} | x_0)\]
    \begin{itemize}
        \item Map the center words into the n-dimensional space 
        \item For the i-th compute the score for a word occupying that position
        \item Normalize the score for each position to get a probability
        \begin{itemize}
            \item Problem on the loss function since the sum requires us to iterate over the entire vocabulary for each example.
            \item The complexity comes from the output layer where we apply the softmax function to get the probability tat each vocabulary word appear in the neighbour of the input word
            \item Convert the problem into a binary classification problem saying if two words are neighbour or not
            \item To train this we have also to sample some negative example, that's why it is called negative sampling
            \item \textbf{Negative Sampling}
            \begin{itemize}
                \item Predicting neighbouring word
                \item Each training sample to update only a small percentage of weights in the word embedding matrix 
            \end{itemize}
            \item \textbf{Subsampling frequent word:} to decrease the number of training examples, clean the dataset
        \end{itemize}
    \end{itemize}
\end{itemize}

\section{Matrix Factorization Models}
\begin{itemize}
    \item Co-occurrences can be computed in two way:
    \begin{itemize}
        \item \textbf{Full document} co-occurrences
        \item \textbf{Windows-based} co-occurrences, capture both semantic and syntactic information
    \end{itemize}
    \item \textbf{Problem} with co-occurrences matrix
    \begin{itemize}
        \item Increase in size with vocabulary
        \item Very high dimensional 
        \item Models are less robust
    \end{itemize}
    \item Idea: store the important info in a fixed small number of dimension: dense vector
    \item \textbf{Singular Value Decomposition}
    \begin{itemize}
        \item Approximate the full matrix by only considering the leftmost k terms in the diagonal matrix (the k largest singular values)
        \item Explain matrix computation
    \end{itemize}
    \item \textbf{Latent Semantic Analysis}
    \begin{itemize}
        \item Assume that words that are close in meaning will occur in similar pieces of text
        \item U-matrix maps words to k-concepts
        \item V-matrix maps k-concepts to documents
    \end{itemize}
    \item \textbf{Point Wise Mutual Information}
    \begin{itemize}
        \item How to represent co-occurrence of words and contexts
        \item Binary feature or frequency count
        \item Sometimes low co-occurrences are very informative and high co-occurrences are not
        \item We need to identify when co-occurrences are higher than we would expect by chance
        \item \textbf{Point wise mutual information}
        \item Problem 
        \item Positive point wise mutual information
    \end{itemize}
    \item \textbf{Brown Cluster}
    \begin{itemize}
        \item Input: large corpus of words
        \item Otput 1: a partition of words into words clusters
        \item Output 2 (general): a hiererchical word clustering
        \item Intuition: similar word appear in similar context. Similar word have similar distribution of words to their immediate left and right
        \item Formualtion of \(Quality(C)\)
        \item \textbf{First Algorithm}
        \begin{itemize}
            \item Start with \(|V|\) clusters 
            \item We want to get k final clusters
            \item We run \(|V| - k\) merge steps
            \begin{itemize}
                \item Pick two clusters \(c_i\) and \(c_j\) and merge them into a single cluster
                \item Continue piking two clusters st the Quality(C) after the merge step is maximized 
            \end{itemize}
        \end{itemize}
        \item \textbf{Second Algorithm}
        \begin{itemize}
            \item Take the top m most frequent words, put each into its own cluster \(c_1, c_2, ..., c_m\)
            \item For \(i = (m  + 1) ... |V|\)
            \begin{itemize}
                \item Create a cluster \(c_{m+1}\) for the i'-th most frequent word. Now we have \(m + 1\) clusters
                \item Choose two clusters from \(c_1, ..., c_{m + 1}\) to be merged, pick the merge that gives a maximum value for Quality(C). Now we are back to m clusters
            \end{itemize}
        \end{itemize}
    \end{itemize}
    \item \textbf{Word2Vec SkipGram Negative Sampling - PPMI Matrix Formulation}
    \begin{itemize}
        \item Skipgram negative sampling is implicitly factorizing a specific matrix of this kind
        \item Two points to note:
        \begin{enumerate}
            \item The entries in the matrix are a shifted positively pointwise mutual information between the word and its context word
            \item The matrix factorization method is not truncated SVD, it minimize the objective function to compute the factorized matricies
        \end{enumerate}
    \end{itemize}
    \item \textbf{Global Vector (GloVE)}
    \begin{itemize}
        \item Pick the best of cont-based and prediction-based models
        \item It construct an explicit word-context matrix
        \item The optimization objective is weighted least-squares loss assigning more weight to the correct reconstruction of frequent items
        \item When using the same word and context vocabularies, the GloVE model represent each word as the sum of it corresponding word and context embedding vectors.
        \item Given two words i and j that occour in text, their co-occurrence probability is defined as the probability of seeing word i in the context of word j
        \item Formal notation
    \end{itemize}
    \item \textbf{GloVe VS SkipGram}
    \begin{itemize}
        \item Skipgram: capture co-occurrence one window at a time
        \item GloVE: captures the count of the overall statistics of how often words appear.
        \begin{itemize}
            \item Appropiate scaling and objective gives count-based models the proprieties and performnce of predict-based models
        \end{itemize}
    \end{itemize}
    \item \textbf{FastText}
    \begin{itemize}
        \item \textbf{Limitation of word2vec}
        \begin{itemize}
            \item Out of vocabulary words: an embedding is created for each word, hence it can't handle words outside its training set.
            \item Morphology: like "eat" and "eaten" are different
        \end{itemize}
        \item Each word w is represented as a bag of character n-Gram
        \item Add \(< >\) to indicate the beginning and the end of words
        \item Include the word w itself in the set of n-Gram
        \item \textit{Heuristics}:
        \begin{itemize}
            \item N-Gram between 3 and 6 characters
            \item Short n-Gram(n = 4): capture syntactic info
            \item Long n-Gram(n = 6): capture semantic info
        \end{itemize}
        \item \textbf{Loss discussion}
        \item Sharing representation across words allowing to learn reliable representation of rare words
    \end{itemize}
\end{itemize}

\chapter{RNN}
\begin{itemize}
    \item \(A_t = \theta_c A_{t-1} + \theta_xx_t \quad h_t = \theta_nA_t\)
    \item Type of RNN
\end{itemize}
\section{Back Propagation Through Time}
\begin{itemize}
    \item Parameters are shared and the derivative are accumulated
    \item The weights of the unrolled RNN are the same
\end{itemize}
\section{RNN and long term dependencies}
\begin{itemize}
    \item The closer I move to the end of the sequence the more I have to add and multiply the weights. During training the gradient could vanish or explode, I will not learn long term dependencies
    \item If weights are large: exploading gradient
    \item If the weights are small: vanishing gradient
    \item \textbf{Fixing Exploading Gradients}
    \begin{itemize}
        \item \textbf{Gradient Clipping}
        \item \textbf{Truncated BPTT}
    \end{itemize}
    \item \textbf{Fixing Vanishing Gradient: adding non-linearity:} \(thanh\), like doing an additional power iteration
    \item RNN exmple: sentiment analysis
    \item RNN as LM
\end{itemize}

\section{Long - Short Term Memory Networks}
\begin{itemize}
    \item LSTM were created with the following goals:
    \begin{itemize}
        \item Long-term dependencies
        \item Incorporation of feedback connections
        \item Preserving valuable info from prior data
        \item Proficiency in handling data sequences
    \end{itemize}
    \item The output of an LSTM depends on
    \begin{itemize}
        \item Current long-term memory, cell state
        \item The previous hidden state, short term memory
        \item Input data
    \end{itemize}
    \item Three gates that regulates the amount of long and short term memories allowed
    \begin{itemize}
        \item \textbf{Forget Gate}: how much of the long-term memory will be remember
        \begin{itemize}
            \item How much we will preserve from the long-term memory
            \item Weight removal of part of the long-term memory
        \end{itemize}
        \item \textbf{Input Gate}: how we update the long-term memory
        \begin{itemize}
            \item \% memory to remember
            \item Potential memory to remember based on short term and input data
            \item How much of the new memory will be added to the long-term memory 
        \end{itemize}
        \item \textbf{Output Gate}: update the short-term memory
        \begin{itemize}
            \item Potential long-term memory to become sort
            \item \% of short to remember
            \item new short memory
        \end{itemize}
    \end{itemize}
    \item Symbols meaning:
    \begin{itemize}
        \item \(tanh \rightarrow\) potential amount of memory to retain 
        \item \(\sigma \rightarrow\) percentage of memory to remember/become 
    \end{itemize}
\end{itemize}

\section{Gated Recurrent Unit}
\begin{itemize}
    \item Combines forget and input gates, into update gate
    \item Merges the cell state and the hidden state
    \item Has fewer parameters than LSTM
    \item No need cell layer
    \item Calculations within each iteration ensure that \(h_t\) values being passed along, either retain a high amount of old information or are jump-started with a high amount of new information
    \item \textbf{GRU VS LSTM}
    \begin{itemize}
        \item GRU significantly fewer parameters and train faster
    \end{itemize}
\end{itemize}

\section{Bidirectional RNN and LSTM}
\subsection{BI-RNN}
\begin{itemize}
    \item Combine two RNN:
    \begin{itemize}
        \item First move forward through time from the start of the sequence 
        \item Second move backword through time from the end of the sequence
    \end{itemize}
    \item Two RNN stacked on top of each other
    \item The one that process the input in its original order and the one that process reversed input sequence
    \item Output computed based on the hidden state of both RNNs
    \item The weight of the forward are different from the backword RNN
\end{itemize}

\subsection{BI-LSTM}
\begin{itemize}
    \item Inputs flows in both directions, capable of using both sides, a side for LSTM
    \item \textbf{Output} is combined in several ways: \textit{average, sum, concatenation, multiplication} 
    \item Every component has information from both past and present
    \item Bi-LSTM can produce more meaningful output
    \item Much slower model
    \item Reqire more time for training
\end{itemize}

\section{Sequence 2 Sequence}
\begin{itemize}
    \item Used to convert one sequence into an other
    \begin{itemize}
        \item Sequences might have different length
        \item Based on encoder-decorder architecture
        \item Combines two RNNs
    \end{itemize}
    \item \textbf{Given an input/source sentence}
    \begin{itemize}
        \item Start and end token of the sentence
        \item Each word to the embedding layer and then into the encoder
    \end{itemize}
    \item \textbf{At each time step:}
    \begin{itemize}
        \item Input o the encoder is the previous hidden state and the embedding of the current word
        \item Encoder output a hidden state which is the vector representation of the sentence so far
        \item Once the final word \(x_t\) has been passed into the RNN, we use the final hidden state ans context vector
    \end{itemize}
    \item \textbf{Now turn of the encoder}: append the \(<SOS>\) token to the start target sequence
    \item \textbf{At each time step:}
    \begin{itemize}
        \item Input: embedding of the current token and the previous hidden state
        \item Initial decoder hidden state is the context vector
    \end{itemize}
    \item \textbf{Points worth nothing:}
    \begin{itemize}
        \item Input/source encoder embedding and output/target decoder embedding are two different embedding layer with different parameters
        \item Linear layer to convert decoder's hidden state into one of the possible words by softmax
        \item The predicted word is used as input for the next loop
    \end{itemize}
    \item \textbf{Training:} once we have our predictor target sequence we compare with the actual target, compute the loss, back-propagation
    \item \textbf{How to select the right word during training}
    \begin{itemize}
        \item Gready search: highest softmax probability
        \item Exhaustive search: generate all possible sequence
        \item Teacher forcing: Use ground truth as input instead of the model output from a prior time stamp
        \item Beam Search: on each step of decoder keep track of the k-most probable partial translations
    \end{itemize}
\end{itemize}

\chapter{Self-Supervised Learning}
\begin{itemize}
    \item Acquiring large scale dataset with curated humman-annotations is hard and expensive
    \item The set of categories to learn need to be prefedined
    \item \textbf{Self-Supervised Learning (SSL)}: a smart way of doing supervised learning, the data and not the human provides annotations
    \item We am to learn good data representation
    \item Need an automated way to obtain a training signal from the data itself, without human supervision
    \item One solution is self-prediction: try to predict the properties of the data or hide part of the data and try to reconstruct it
    \item Pretext and downstream tasks
    \begin{itemize}
        \item Define a pretext task and train the network for that
        \item These features are used for a different downstream task where labels are available
    \end{itemize}
    \item Self-Supervised Learning in NLP:
    \begin{itemize}
        \item Center word prediction
        \item Neighbour word prediction
        \item Neighbour sentence prediction
        \item Auto-regressive modelling
        \item Masked language modelling
        \item Next sentence prediction
        \item Sentence order prediction
        \item Sentence permutation
        \item Document rotation
    \end{itemize}
\end{itemize}


\chapter{Attention Mechanism}
\section{Neural Turing Machine}
\begin{itemize}
    \item Goal: taking input and output and learn algorithms that map from one to the other
    \item We feed random input and the corresponding expected outputs from the algorithms we intend to learn
    \item NTM learns to read and write data from the external memory at different time steps to some a given task
    \item \textbf{Controller:} responsible for making the interface between the:
    \begin{itemize}
        \item Input sequence
        \item Output representation
        \item Memory through read and write heads
    \end{itemize}
    \item \textbf{Memory}; contains vectors
    \item NTM lears
    \begin{itemize}
        \item When to write to memory
        \item Which memory cell to read from
        \item How to convert result of read into final output
        \item When to stop writing: avoiding drifting
    \end{itemize}
    \item \textbf{In a NTM:}
    \begin{itemize}
        \item Controller: LSTM or RNN
        \item The hidden state to the read heads
        \item Read head retrieve the information from the memory
        \item The retrieve data is passed to the controller again
        \item The controller send the hidden state to the write head that decide what to store
    \end{itemize}
    \item \textbf{Direct access is not differentiable, so we use attention}
    \begin{itemize}
        \item I'm not directly accessing a memory locations
        \item I'm accessing all memory locations with a certain prob dist that represent the attention scores
        \item Multiply the vector of probability distribution times the matrix that contain all the records
    \end{itemize}
    \item \textbf{Blurry Operations}: interact to a greater or lesser degree with all the elements in memory rather than addressing a single one or few elements directly.
    \begin{itemize}
        \item Degree determined by attention focus 
        \begin{itemize}
            \item Constraints each read and write operation to interact with a small portion of the memory while ignoring the rest
            \item The proportion of memory taken into attentional focus is determined by normalized weights emitted by the heads
        \end{itemize}
    \end{itemize}
    \item \textbf{Memory Addressing}
    \begin{itemize}
        \item \textit{Content Based - Addressing}
        \begin{itemize}
            \item Cosine similarity between the embeddings form the lstm and the embedding in the memory
            \item Softmax between the temperature of a cosine similarity and the other
        \end{itemize}
        \item \textit{Local Based  - Addressing}
        \begin{itemize}
            \item Interpolation
            \begin{itemize}
                \item The new weight account from both the content based addressing and the focus in the last time stamp
            \end{itemize}
            \item Convolutional Shift
            \begin{itemize}
                \item It smoothly shift the weights left to right based on a parameter
                \item This allow NTM to perform basic algorithm like copy and sort
                \item Very close to the head shifting in a classical turing machine
            \end{itemize}
            \item Sharpening
            \begin{itemize}
                \item If we index N memory locations from 0 to N-1, the rotation applied to the weights by the parameter can be expressed as a circular convolution where all index arithmetic are computed modulo N
                \item It blurs out data, how to fix, sharpening, highlight the most relevant
            \end{itemize}
        \end{itemize}
    \end{itemize}
    \item \textbf{Read operations}: is a weighted sum, in the most desiderable situation, the weights are OHE vector, 0s and one 1
    \item \textbf{Write Operation}:
    \begin{itemize}
        \item Combination of erase and add operations
        \item The process is composed with the previous state and new input, Implements a similar mechanism as the forget and input gates in LSTM
    \end{itemize}
    \item \textbf{Controller unrolled}
    \begin{itemize}
        \item Once the head has updated its weight vector it is ready to operate on the memory depending on the head type:
        \begin{itemize}
            \item Read head: it outputs a weighted combination on the memory locations
            \item If it is a write head the content of the memory is modified according to weights with an erase and an add vector
        \end{itemize}
    \end{itemize}
\end{itemize}


\section{Additive Attention}
\begin{itemize}
    \item \textbf{Bottleneck Problem}
    \begin{itemize}
        \item NN needs to be able to compress all the necessarily information of a source sentence into a fixed length vector
        \item Difficult to cope with long sentences especially those that are longer than the sentences in the training corpus 
        \item Attention mechanism helps to look at all hidden states from encoder sequence for making predictions
        \item How we decide which states are more or less useful for every prediction at every time step of the decoder
        \item Use a NN to learn which hidden encoder states to attend to and by how much
    \end{itemize}
    \item \textbf{Alignment Computation}
    \begin{itemize}
        \item How important is the embedding \(h_j\) compared to what I have in the hidden state of the decoder at the previous state
        \item a is a FFNN
    \end{itemize}
    \item \textbf{Attention Component}
    \begin{itemize}
        \item Score converted into a prob dist via softmax
        \item \(a_{ij}\) importance of annotation \(h_j\) wrt the previous hidden state \(s_{i-1}\) in deciding the next state \(s_i\) and generating \(y_i\) 
    \end{itemize}
    \item \textbf{Context vector}
    \begin{itemize}
        \item Depends on a sequence of annotations to which the decoder map an input sequence
        \item Each annotations contains informations about the whole input sequence with a strong focus on the parts surrounding the i-th word of the input sequence 
    \end{itemize}
    \item \textbf{Attention is great indeed it}
    \begin{itemize}
        \item Significantly improves NTM performance
        \item Solves the bottleneck problem
        \item Helps with vanishing gradient problem
        \item Provide some interpretability
    \end{itemize}
    \item \textbf{General DL Framework}
    \begin{itemize}
        \item Given a set of vector values and a vector query, attention is a technique to compute a weighted sum of the values dependent on the query
    \end{itemize}
    \item \textbf{Intuition}
    \begin{itemize}
        \item Weighted sum is a selective summary of the information contained in the values, where the query determines which values to focus on
        \item Is a way to obtain a fixed size representation of an arbitrary set of representations
    \end{itemize}
    \item \textbf{RNN Problems}
    \begin{itemize}
        \item Hard to learn long distance dependencies
        \item Linear order of words
        \item Embedding combined with all the previous one and at a certain point we completely loose information of the initial input sequence
        \item Forward and backword passes have un-parallelizable operations
        \begin{itemize}
            \item Future RNN hidden states can't be computed in full before past hidden states have been computed
        \end{itemize}
    \end{itemize}
\end{itemize}


\subsection{Self-Attention}
\begin{itemize}
    \item Attention treats each word's embedding representation as a query to access and incorporate information from a set of values
    \item \textbf{Attention as a soft-lookup method}
    \begin{itemize}
        \item Attention as performing fuzzy lookup in a key value store
        \item In a look-up table we have a table of keys that map to values. The query matches one of the keys returning its value
        \item In attention the query matches all keys softly, to a weighted between 0 and 1. The keys values are multiplied by the weights and summed
    \end{itemize}
    \item \textbf{Self-Attention Mechanism on the same sequence}
    \begin{itemize}
        \item Attention operates on queries, keys and values
        \item In self-attention the queries, keys and values are drawn from the same source
        \item Formulas
    \end{itemize}
\end{itemize}


\subsection{Attention Mechanism}
\begin{itemize}
    \item Self attention, can we use it as our main building block? Can it be a drop-in replacement for recurrence?
    \item NO! it has few issues
    \begin{enumerate}
        \item \textbf{Doesn't have inherit notion of order}
        \begin{itemize}
            \item \textbf{Solution}: add positional representation to the inputs
            \item We need to encode the order of the sentence in our keys, queries and values
            \item \textbf{Sinusoidal Representation:} concatenate sinusoidal functions of varying periods
            \begin{itemize}
                \item Why:
                \begin{itemize}
                    \item Periodicity
                    \item Constrained values
                    \item Extrapolation for long sequences
                \end{itemize}
                \item Pros: periodicity and extrapolation
                \item Cons: not learnable, extrapolation doesn't really work
            \end{itemize}
            \item \textbf{Learned absolute position representation}
            \begin{itemize}
                \item Let all \(p_i\) be learnable parameters
                \item Pro: flexibility: each position gets to be learned to fit the data
                \item Con: definitely can't extrapolate to indices outside \(1,...,T\)
            \end{itemize}
        \end{itemize}
        \item \textbf{No non-linearities for DL. It's all just weighted average}
        \begin{itemize}
            \item \textbf{Solution}: easy fix: apply the same FFNN to each self-attention output
            \item So instead of passing directly the soft-attention result to the next step we send it in a FFNN
        \end{itemize}
        \item \textbf{Need to ensure we don't "look at the future" when predicting a sequence}
        \begin{itemize}
            \item \textbf{Solution}: Masking
            \begin{itemize}
                \item Parallelize operations while not looking at the future
                \item Keeps info about the future from \textit{"leaking"} to the past
            \end{itemize}
            \item Self attention in decoders, we need to ensure we can't peek at the future
            \item At every timestamp we could change the set of keys and queries to include only past words
            \item Enable parllelization to future words by setting attention scores to \(-\infty\)
        \end{itemize}
    \end{enumerate}
\end{itemize}



\chapter{Transformers}
It is an \textbf{encoder - decoder} model with attention and without recursion.
\section{Encoder Part}
\begin{itemize}
    \item \textbf{Input Embedding}: convert a token into vector
    \item \textbf{Positional Encoding}: add location of the word within the sentence via sinusoidal representation
    \item \textbf{Multi-head Attention}
    \begin{itemize}
        \item Self-attention, keys, queries and vales comes from the same source
        \item Query - key dot product in one matrix multiplication \(XQ(XK)^T \in \mathbb{R}^{n \times n}\)
        \item Softmax and compute the weighted average with an other matrix multiplication
        \[softmax(XQ(XK)^T) \cdot XV  \in \mathbb{R}^{n \times d}\]
        \item What if we want to look in multiple places in the sentence at once
        \begin{itemize}
            \item We define multiple attention heads through multiple Q,K,V matrcies, h for example
            \item Each attention head perform attention independently
            \item Then outputs of all heads are combined
            \item Each head gets to "look" at different things, and construct value vectors differently
            \item It's not really more costly, explain
        \end{itemize}
    \end{itemize}
    \item \textbf{Add \& Norm}: scaled product attention is a final variation to aid transformer training
    \begin{itemize}
        \item When dimensionality becomes large, dot products between vectors tend to become large
        \item We define the attention scores by \(\sqrt{\frac{d}{h}}\) to stop the scores becoming large just as a function of \(\frac{d}{h}\)
    \end{itemize}
    \item \textbf{Encoder Tricks}
    \begin{itemize}
        \item \textbf{Residual Connections}
        \begin{itemize}
            \item Help the model to train better
            \item Though to make the loss landscape considerably smoother
        \end{itemize}
        \item \textbf{Layer Normalization}
        \begin{itemize}
            \item Help the model to train faster
            \item Cut down uninformative variations in hidden vectors values by normalizing to unit mean and sd within each layer
            \item Z score (how many sd our point is far from the mean) \(\times\) gain + bias
        \end{itemize}
    \end{itemize}
\end{itemize}

\section{Decoder Part}
\begin{itemize}
    \item \textbf{Cross-Attention}
    \begin{itemize}
        \item Keys and values are drawn from the encoder
        \item Queries are drawn from the decoder
        \item \(H\) vector of concatenation of encoder output
        \item \(Z\) vector of concatenation of input decoder
        \item Query and \(Z\) times Keys and \(H\): \(ZQK^TH^T  \in \mathbb{R}^{T \times T}\)
        \item Softmax and compute the weighted average with \(H\) and Values
        \[softmax(ZQK^TH^T) \cdot HV \in \mathbb{R}^{T \times d}\]
    \end{itemize}
    \item \textbf{Masked Self-Attention}
    \begin{itemize}
        \item We have \(h\) attention mask for \(h\) heads
    \end{itemize}
\end{itemize}

\section{Transformer Drawbacks}
\begin{itemize}
    \item \textbf{Positional Representation}
    \item \textbf{Quadratic Compute in Self-Attention}
    \begin{itemize}
        \item Highly parallelizable
        \item However the total \# operations grows as \(O(T^2d)\), where \(T\) is the sequence length and \(d\) is the dictionary
    \end{itemize}
\end{itemize}

\chapter{Contextualized Word Embedding}
\section{ELMo}
Replace static Embeddings with context-dependent embeddings
\begin{itemize}
    \item Each token's representation is a function of the entire input sequence computed by a deep bidirectional LM
    \item Return for each token a linear combination of its representation across layers
    \item Different layer capture different information
    \item Each layer of this language model computes a vector representation for each token
    \item For each task: train task dependent softmax weights to combine the layer wise representation into a single vector for each token jointly with a task specific mode that uses those vectors
    \item \textit{Forward LM:} deep LSTM that goes form the start to the end to predict the token based on the prefix
    \item \textit{BackwordLM:} deep LSTM that goes form the end to the start to predict the token based on the suffix
    \item Train these LMs jointly with the same parameters for the token representation ad the softmaz layer
    \item How to use it
    \begin{itemize}
        \item Embedding into choosen architecture
        \item Frozen Embeddings
        \item Fine Fine-Tuning, bee careful for the catastrophic forgetting
    \end{itemize}
\end{itemize}

\section{BERT}
\textbf{Bidirectional Encoder Representation from Transformer}
\begin{itemize}
    \item Fed the transformer encoder with a sequence
    \item Get the embeddings for each words
    \item \textbf{Goal} Understand laguage
    \item 12 or 24 encoders layer (Small or Large)
    \item Pre-Training phase:
    \begin{itemize}
        \item Pretrain to understand the language
        \begin{itemize}
            \item masked Language Models (MLM)
            \item Next Sentence Predictions (NSP) 
            \item Both tasks are jointly trained
        \end{itemize}
        \item Finetune on specific tasks
        \begin{itemize}
            \item Softmax on final layer of [CLS] token
            \item Softmax to the tokens in the sequences
            \item [CLS] question [SEP] answer passage [SEP], lean to predict start and end token on answer token
        \end{itemize}
    \end{itemize}
    \item Explain training-eval phase
    \item Mask out k\% of the input words and then predict the masked words, k = 15 usually
    \begin{itemize}
        \item Too little masking: too expensive to train
        \item Too much masking: not enought context 
    \end{itemize}
    \item Problem: mask token never seen at fine-tuning
    \begin{itemize}
        \item 15 \% of words to predict, we replace them with
        \item 80 \% with [mask]
        \item 10 \% random word
        \item 10 \% keep the same
    \end{itemize}
\end{itemize}

\section{GPT}
\textbf{GPT: Generative Pre-Trained Transformer}
\begin{itemize}
    \item Feed the transformer decoder with a sentence predict the next word
    \item Only a single Masked Multi-Head Attention
    \item Language model
\end{itemize}
\subsection{GPT-1}
\begin{itemize}
    \item Auto-Regressive 12 layer transformer decoder
    \item Pre-trained on raw text as language model
    \item Fine-tuned on labeled data: classification, entailment, similarity, multiple choice
\end{itemize}
\subsection{Tokenization}
\begin{itemize}
    \item What happens when we encounter a word at test time that we're never seen in our training data?
    \begin{itemize}
        \item in Word level tokenization no way of assigning index, no word emebdding, can't process that input sequence
    \end{itemize}
    \item Solution: low-frequency word in training with a special \(<UNK>\) token, used to handle unseen words
    \item Problem: unseen word are all the same
    \item Word-level tokenization treats different forms of the same word
    \begin{itemize}
        \item Open, opened, opens
        \item Separate types and embeddings
        \item This can be problematic when training over smaller dataset
        \item Re-learn the meaning of words
    \end{itemize}
\end{itemize}
\subsection{Byte Paring Encoding}
\begin{itemize}
    \item From base vocabulary
    \item Count up frequency of each character pair in the data, choose the most occured
    \item Select it and merge the characters together into one symbol, Add this new symbol to the vocab then retokenize the data
    \item Keep repeating this process, untill we stop
\end{itemize}
\subsection{Word Piece Encoding}
\begin{itemize}
    \item Good balance between flexibility of single characters and the efficiency of full words for decoding, sidestep the needs of special treatment of unknown words
    \item Pretrained model performs the word segmentation
    \item Special encoding for out-of-vocabulary words, explain
\end{itemize}
\subsection{GPT-2}
\begin{itemize}
    \item Larger, more parameters, more vocabularies
    \item \textbf{Emergent zero-shot learning}
    \begin{itemize}
        \item Ability to do many tasks with no examples and no gradient updates
        \item Specifying the right sequence prediction problem
        \item Comparing probabilities of sequences
        \item We can get interesting zero shot learning behaviour if we are creative enough with how we specify our task
    \end{itemize}
\end{itemize}
\subsection{GPT-3}
\begin{itemize}
    \item Specify a task by simply prepending example of the task before our example
    \item Called also in-context learning, to stress that no gradients are performed when learning a new task
    \item What are the limits of promting
    \begin{itemize}
        \item Some tasks seems too hard even for LMs to learn through promting alone, like math operations
        \item Solution: \textbf{Chain-of-Though Promting}: helps the system explaining to get the answer
    \end{itemize}
\end{itemize}
\subsection{RoBERTa}
Same model as BERT but better tuned
\begin{itemize}
    \item Better HP
    \item Larger model
    \item Big bathces with more data
    \item Removing NSP
    \item Trained on longer sequence
    \item \textbf{Static Masking (BERT)}
    \begin{itemize}
        \item Precompute the mask before training
        \item Masks are constant at each iteration
    \end{itemize}
    \item \textbf{VS Dynamic Masking (RoBERTa)}
    \begin{itemize}
        \item Change mask on the fly
        \item Change it at every training iteration, results in a more data and generalization capabilities
    \end{itemize}
\end{itemize}
\subsection{BART}
\begin{itemize}
    \item BERT: encoder, good for analysis tasks, trained with MLM 
    \item GPT: decoder, good for LM
    \item \textbf{BART}: both encoder and decoder
    \item \textbf{Pre-Training Tasks}
    \begin{itemize}
        \item Token masking, deletion, filling
        \item Sentence permutation
        \item Text infilling
        \item Document rotation
    \end{itemize}
\end{itemize}
\subsection{T5}
\begin{itemize}
    \item Treat every NLP tasks as a "txt-to-text" problem, same model, same HP, same loss function
    \item How: adding a task specific prefix to the input sequence, and pretrain the model to get task specific output 
    \item T5 uses an encoder decoder arch. with the differences:
    \begin{itemize}
        \item Layernorm applied before each attention and FF transformation
        \item No additive baias is used for layer norm
        \item A simpler positional embeddings is used, it uses a scalar to the corresponding logit used to compute attention weights
        \item Dropout is applied through the network
    \end{itemize}
\end{itemize}
\subsection{GPT-3.5}
\begin{itemize}
    \item LM \(\neq\) Assisting users
    \item Pre-training can improve NLP app by serving as parameter initialisation
    \item From LM to assistant
    \begin{itemize}
        \item \textbf{Instruction Fine-tuning}: collect examples of pair across many tasks and fine-tune a LM, but has a lot of limitations:
        \begin{itemize}
            \item Expensive to collect ground truth
            \item P1: task like open-encoded creative generation has no right answer
            \item P2: LM penalize all token-level mistakes equally but some errors are worse than other
            \item Miss-match between LM objective and human preferences
        \end{itemize}
        \item \textbf{Reinforcement Learning for Human Feedback (RLHF)}
        \begin{itemize}
            \item Training a LM on some task
            \item Obtain human reward, higher is better
            \item We want to maximise the reward of samples from our LM
            \item Gradient ascend
            \item Reward is not differentiable
            \item Policy gradient, tool for estimating and optimizing this objective
            \item Log derivate trick
            \item Plug all back
            \item Put the gradient inside the expectation, we can approximate this objective with montecarlo samples, performing m-samples and then compute the average
            \item I have the gradient of my model's parameter wrt the samples, I can get an update rule so gradient Ascent
            \begin{itemize}
                \item If R is ++: gradient step to maximise the sample prob
                \item If R is --: gradient step to minimise the sample prob
            \end{itemize}
            \item Now for any arbitrary non differentiable reward function R(s) we can train our LM to maximise the expected reward
            \item \textit{Problem 1}: human in the loop is expensive, solution model their preferences as a separate NLP problem
            \item \textit{Problem 2}: human judgements are noisy and misscalibrated, solution ask for pairwise comparison
        \end{itemize}
    \end{itemize}
\end{itemize}
\subsection{Istruction-GPT}
\begin{itemize}
    \item Collect demonstration data and train a supervised policy
    \item Collect comparison data and train a reward model
    \item Optimize a policy against the reward model using reinforcement learning (RLHF)
\end{itemize}

\chapter{Graph and NLP}
\section{Theoretical Foundations of GNN}
\begin{itemize}
    \item Convolutional neural networks respect translational invariance
    \item Patterns are interesting irrespective of where they are in the image
    \item \textbf{Locality}: neighbouring pixels relate much more strongly than distant ones
    \item What about arbitrary graphs?
    \begin{itemize}
        \item The nodes of a graph are not assumed to be in any order 
        \item We would like to get the same results for two isomorphic graphs
        \item \textbf{Permutation invariance and equivariance}
    \end{itemize}
    \item Assuming the graph has no edges
    \item \(x_i \in \mathbb{R}\) be the feature of node i
    \[X = (x_1, ..., x_n)^T\]
    \item We have specified a node ordering!
    \item We would like the result of any neural networks to not depend on this
    \item It will be useful to think about the operations that change the node order, like \textbf{Permutations}
    \item Each permutation defines an \(n \times n\) matrix, row with one 1 and all zeros, col i (which is referring to the row) goes to the position in wich is present the 1
    \item \textbf{Permutation Invariance}
    \begin{itemize}
        \item Design functions \(f(X)\) over sets that will not depend on the order
        \item Thus applying a permutation matrix shouldn’t modify the result!
        \item \(f(X)\) is permutation invariant if, for all permutation matrices \(P\) st: 
        \[f(PX) = f(x)\]
        \item Generic form description (see notes)
    \end{itemize}
    \item \textbf{Permutation Equivariance}
    \begin{itemize}
        \item We want to still be able to identify node outputs, which a permutation-invariant aggregator would destroy
        \item Seek functions that don’t change the node order
        \item \(f(X)\) is permutation equivariant if, for all permutation 
        matrices \(P\)
        \[f(PX) = Pf(X)\]
        \item Each node’s row is unchanged by \(f\)
        \item Equivariant set functions as transforming each node input \(x_i\) into a latent vector \(h\)
    \end{itemize}
    \item We can represent these edges with an adjacency matrix, \(A\)
    \item Node permutations now also accordingly act on the edges
    \item We need to appropriately permute both rows and columns of \(A\), When applying a permutation matrix \(P\), this amounts to \(PAP^T\)
    \[\text{Invariance} \quad f(PX) = f(X) \quad f(PX, PAP^T) = f(X,A)\]
    \[\text{Equivariance} \quad f(PX) = Pf(X) \quad f(PX, PAP^T) = Pf(X,A)\]
    \item On sets, we enforced equivariance by applying functions to every node in isolation
    \item \textbf{Node’s neighbourhood}
    \begin{itemize}
        \item (1-hop) neighbourhood
        $N_i = \{j : (i,j) \in \epsilon \land (j, i) \in \epsilon \}$
        \item $X_{N_i} = \{x_j : \in N_i\}$
    \end{itemize}
    \item \textbf{Permutation Equivariance Fucntions}
\end{itemize}

\section{Convolutional GNN}
\begin{itemize}
    \item The goal is to learn a function of signals/features on a graph \(G=(V,E)\) which takes as input:
    \begin{itemize}
        \item A feature description \(x_i\) for every node \(i\) summarized in a \(N \times D\) feature matrix \(X\)
        \item A representative description of the graph structure in matrix form, adjacency matrix \(A\)
        \item And produces a node-level output \(Z\)
    \end{itemize}
    \item \textbf{Limitations}
    \begin{itemize}
        \item Initial node features are not considered
        \begin{itemize}
            \item Add self-loops in \(A\)!
        \end{itemize}
        \item The matrix \(A\) is not normalized, hence using it in the multiplication will change the scale
        \begin{itemize}
            \item Use the normalized \(A\)!
        \end{itemize}
        \item \textbf{These two fixes will lead to the GCN formulation}
    \end{itemize}
\end{itemize}

\section{Graph Attention Networks}
\begin{itemize}
    \item Features of neighbours aggregated with implicit weights (via attention)
    \item Useful as “middle ground” w.r.t. capacity and scale
    \item Attention coefficients \(a_{ij}\) implicitly defined via self-attention over node features
    \item Alpha is the attention mechanism
    \item Alpha’s parameters trained jointly with the remaining architecture
    \item Works in both inductive and transductive  settings
    \item Attention Score 
    \item Multi-Head Attention
\end{itemize}



\section{Message Passing Neural Network}
\begin{itemize}
    \item Compute arbitrary vectors (“messages”) to be sent across edges
    \item Two Phases
    \begin{itemize}
        \item Message Passing
        \item Readout
    \end{itemize}
    \item \textbf{Extending Message Passing Framework}
    \begin{itemize}
        \item What about edges?
        \item Compare different types of inductive biases
        \item Propose a general and comprehensive GN formulation
    \end{itemize}
    \item \textbf{Type of Baiases}
    \begin{itemize}
        \item Fully Connected
        \item Convolutional
        \item Recurrent
        \item Graph Network
    \end{itemize}
    \item \textbf{GNN: a General Formualtion}
\end{itemize}

\section{Graph (NN) in NLP}
\begin{itemize}
    \item Represent natural language as a graph
    \begin{itemize}
        \item Dependency graphs
        \item Text graph containing multiple hierarchies of elements
    \end{itemize}
    \item \textbf{Graph Construction from Text}
    \begin{itemize}
        \item Different NLP tasks require different aspects of text
        \item Static vs dynamic construction
        \item Goal: good downstream task performance
    \end{itemize}
    \item \textbf{Static Graph Construction}
    \begin{itemize}
        \item Problem Setting
        \begin{itemize}
            \item Input: raw text
            \item Output: graph
        \end{itemize}
        \item Graph structure learned during preprocessing by argumenting text with domain knowledge
        \item \textbf{Types:}
        \begin{itemize}
            \item Dependency Graph
            \item Constituency Graph
            \item Abstract Meaning Representation Graph
            \item Information Extraction (IE) Graph
            \item Knowledge Graphs Q\&A
            \item Co-occurrence Graph 
            \item SQL Graph
            \item Dynamic Graph Construction
        \end{itemize}
    \end{itemize}
    
    \item \textbf{Dynamic Graph Construction}
    \begin{itemize}
    \item Problem Setting:
    \begin{itemize}
        \item Input: raw text
        \item Output: graph    
    \end{itemize}
    \item Graph structure (adjacency matrix) learning on the fly, joint with graph representation learning
    \end{itemize}
\end{itemize}

\section{Knowledge Graphs in NLP}
\begin{itemize}
    \item \textbf{Tasks}
    \begin{itemize}
        \item Link prediction / triple classification
        \begin{itemize}
            \item Link prediction
            \begin{itemize}
                \item Learning to rank problem
                \item Information retrevial metrics
                \item No ground truth negativities
            \end{itemize}
            \item Triple Classification
            \begin{itemize}
                \item Binary classification task and metrics
                \item Test set require positive and ground truth negatives
            \end{itemize}
        \end{itemize}
        \item Collective node classification/ link-based clustering
        \item Entity matching
    \end{itemize}
    \item \textbf{Transductive Link Prediction}: When both subject and object off the predicted link occur in the training knowledge graph
    \item \textbf{Inductive Link Prediction}: when subject or object of the predicted link do not occur in the training knowledge graph
    \item \textbf{Representation Learning}
    \begin{itemize}
        \item Learning representation of nodes and edges
        \item Our goal is Knowledge Graph Embeddings
        \begin{itemize}
            \item Automatic, supervised learning of embeddings, projections of entities and relations into a continuous low-dimensional space
            \item Scoring Function: \(f\) assign a score to a triple \((s,p,o)\), high score = triple is very likely to be factually correct
            \item Translation-based Scoring Functions: \(f_{TransE} = ||(e_s + r_p) - e_o ||_n\)
            \item Loss function
            \begin{itemize}
                \item Pairwise Margin-Based Hinge loss
                \item Negative Log-Likehood / Cross Entropy
            \end{itemize}
        \end{itemize}
    \end{itemize}
\end{itemize}

\section{Knowledge Graphs (NN)}
Multi Relational Graphs, see notes.

\chapter{Vision and Language}
\section{ViT}
\textbf{ViT: Vision Transformer}
\begin{itemize}
    \item N input patches each shape for example 3x16x16
    \item Linear projection to D-dimensional vector
    \item Add positional embedding: learn D-dim vector per position and a special extra input, the classification token like the [CLS] token
    \item Transformer Encoder, like NLP Transformer
    of\item Output vectors, last output is the linear projection to C-dim vector o predicted class scores
    \item \textbf{ViT vs CNN}
    \begin{itemize}
        \item IN most CNNs decrease resolution and increase channels as you go deeper in the network \textit{(hierarchical architecture)}
        \item In ViT all blocks have the same resolution and number of channels \textit{(isotopic architecture)}
    \end{itemize}
    \item \textbf{Pros}:
    \begin{itemize}
        \item Can learn global features of images
        \item ViTs are not as sensitive to data argumentation as CNNs, thus they can be trained on smaller dataset
        \item ViTs can be used for a variety of image classification tasks
    \end{itemize}
    \item \textbf{Cons}:
    \begin{itemize}
        \item Computationally expensive, large number of parameters
        \item Are not efficients as CNNs at processing images since they need to attend to every part of the image, even if it is not important for the task
    \end{itemize}
\end{itemize}


\section{CLIP}
\textbf{CLIP: Contrastive Language-Image Pre-Trained}
\begin{itemize}
    \item Motivation: instead of using a set of labels, get supervision from NL
    \item Result: robust zero-shot inference, multimodel feature space
    \item Series of text description \(\rightarrow\) textual embedding
    \item Series of images \(\rightarrow\) vision emebdding
    \item Maximise the diagonal similarities, minimize the rest
    \item Compute the cosine similarities of these embedding scaled by a temperature parameter and normalized into a prob dist via softmax
    \item Use symmetric cross entropy loss
    \item \textbf{Zero-shot image Classification}
    \begin{itemize}
        \item Image, encode it into a feature space
        \item Pick all the classes and create a text with prefix: "A photo of a \{class name\}"
        \item Convert the text using text encoder
        \item Compute cosine similarity
        \item Pick the top similar pair of embedding
        \item Zero shot image classification
    \end{itemize}
\end{itemize}

\section{Image Captioning}
\begin{itemize}
    \item Image into a CNN
    \item Output used as context vector to my sequence-to-sequence model (RNN)
    \item Auto-regressive
    \item I continue until I obtain the token \(<END>\)
\end{itemize}


\subsection{Image Captioning using Spartial Features}
\begin{itemize}
    \item \textbf{Problem}: input is "bottleneck" through c, model needs to encode everything it wants to say within c
    \item \textbf{Solution}: Attention idea, new context vector at every time step, each one will attend to different image regions
    \begin{itemize}
        \item Each time step of \textbf{decoder} it uses a different context vector that looks at different parts of the input image
        \item This entire process is differentiable
        \item Model chooses its own weights. No attention supervision is required
    \end{itemize}
\end{itemize}

\subsection{Image Captioning using Transformers}
\begin{itemize}
    \item No recurrence at all
    \item We don't need convolutions
    \item Transformers from pixels to language
\end{itemize}


\section{Diffusion Models}
\begin{itemize}
    \item \textbf{Forward diffusion process}: at each step add some gaussian noise until we destroy the image structure. At each iteration we can decide how much noise we want tot add.
    \item \textbf{Reverse denoising process}: at each step generate a bit cleaner version of the image, gradually reconstruct the image
    \item Sampling from a gaussian dist with mean and std determined by a NN
    \begin{itemize}
        \item Gaussian are convinient since we can jump to arbitrary time step in the process
    \end{itemize}
\end{itemize}

\subsection{DDPM: Denoising Diffusion Probabilistic Models}
\begin{itemize}
    \item \textbf{Challenges}
    \begin{itemize}
        \item Training and sampling in high-res space consume a lot of resources
        \item Images are high-dimensional
    \end{itemize}
    \item \textbf{Remedies}
    \begin{itemize}
        \item Downsampling images and next up-sampling generations
        \item Do diffusion in latent space
        \begin{itemize}
            \item Compressed space so it has lower resolution
            \item It is perceptual compression, model doesn't need to learn imperceptible details
            \item It is a projection onto lower dimensional space
        \end{itemize}
        \item We can condition the LDMs either via concatenation or by a more general cross-attention mechanism
        \item The denoising step is done in the latent space
    \end{itemize}
\end{itemize}



\chapter{NLP Tasks}
\section{Part of speech Tagging}
\begin{itemize}
    \item Input: a sequence of word tokens
    \item Output: a sequence of part-of-speech tags t
    \item \textbf{Why part of speech tagging}
    \begin{itemize}
        \item Can be useful for other NLP tasks: parsing, MT, sentiment, text-to-speech
        \item Linguistic or language-analytic computational tasks
    \end{itemize}
    \item \textbf{A limited number of tags for word class}
    \begin{itemize}
        \item Distributed criteria: same context, same syntatic functions
        \item Morphological criteria 
        \item Not about meaning
    \end{itemize}
    \item \textbf{Open-class parts of speech}
    \begin{itemize}
        \item Nouns
        \item Verbs
        \item Adjective
        \item Adverbs
        \item Preposition
        \item Determiners
        \item Pronouns 
    \end{itemize}
    \item \textbf{Why tagging is hard}
    \begin{itemize}
        \item If every word by spelling was a candidate for just one tag, positional tagging would be trivial
        \item This won't always work
        \begin{itemize}
            \item Roughly 15\% of word types are ambiguous, but those 15\% tend to be very common
            \item So around the 60\% of word tokens are ambiguous
        \end{itemize}
    \end{itemize}
    \item Positional tagging algorithm
    \begin{itemize}
        \item Hidden markov model
        \item RNNs, transformers
        \item LLM
    \end{itemize}
\end{itemize}


\section{Name Entity Recognition}
\begin{itemize}
    \item Anything that can be referred to with a proper name, most common 4 tags:
    \begin{enumerate}
        \item PER: person
        \item LOC: location
        \item ORG: organization
        \item GPE: geo-political entity
    \end{enumerate}
    \item Tings that helps us to understand the context of the text
    \item Find spans of text that constitute proper names, tag the type of the entity
    \item \textbf{Why NER is hard?}
    \begin{itemize}
        \item Segmentation, we have to find and segment entities
        \item Type Ambiguity
    \end{itemize}
    \item \textbf{BIO Tagging}: B \(\rightarrow\) beginning, I \(\rightarrow\) inside, O \(\rightarrow\) outside
\end{itemize}

\section{Co-reference Resolution}
Identify all mentions that refer to the same entity in the text
\subsection{Detect the Mention}
\begin{itemize}
    \item Part of the text that can receive mentioning. Pronouns, named entities, noun phrases
    \item Making all pronouns, named entities and noun phrases as mentions cause over-generation of mentions
    \begin{itemize}
        \item Could train a classifier to filter out spurious mentions
        \item Keep all mentions as "candidate mentions"
        \item We can build a model that begins with all spans and jointly does mention detection and co-reference resolution end-to-end in one model
    \end{itemize}
    \item \textbf{Mention Pairs Model}
    \begin{itemize}
        \item Train a binary classifier that assign every pair of mentions a probability of being co-referent. Positive near 1, negative near 0
    \end{itemize}
    \item \textbf{Mention airs Model at Test Time}
    \begin{itemize}
        \item Co-reference resolution is a clustering task, only scoring pairs, what do
        \item Pick some threshold and add co-reference links between mention pairs where \(p(m_i, m_j)\) is above the threshold
        \item Take transitive closure to get the clustering, add missing link, but not all, explain why
    \end{itemize}
\end{itemize}

\subsection{Clustering Based Approach}
\begin{itemize}
    \item Start with each mention in its own singleton cluster
    \item Merge a pair of clusters at each step. Use a model to score which cluster merges are good
    \item Training phase
    \begin{enumerate}
        \item Produce a vector for each pair of mentions
        \item Poolig operation over matrix of mention pair representation to get cluster-pair representation
        \item Score the candidate cluster, merge by taking the dot product of the representation with a weight vector
    \end{enumerate}
    \item Current candidate cluster depend on the previous one, use reinforcement learning to train the model
    \item \textbf{End-to End Models}
    \begin{itemize}
        \item Use an LSTM, attention
        \item Do mention detection and co-reference end-to-end
        \item No mention detection step, instead consider every span of text as candidate mention
        \item Embed the words in the document using a word embedding matrix and a character level CNN
        \item Run Bi-LSTM over the document
        \item Represent each span of text i going from start(i) to end(i) as a vector
    \end{itemize}
\end{itemize}