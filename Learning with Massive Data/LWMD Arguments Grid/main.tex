\documentclass[12pt,oneside]{report}
\usepackage[utf8]{inputenc}
\usepackage{graphicx}
\usepackage{caption}
\usepackage{subcaption}
\usepackage[a4paper,width=150mm,top=25mm,bottom=25mm,bindingoffset=6mm]{geometry}
\usepackage{fancyhdr}
\usepackage[english]{babel}
\pagestyle{fancy}
\usepackage{biblatex}
\addbibresource{references.bib}
\usepackage{comment}
\usepackage{amsfonts}
\usepackage{tcolorbox}
\usepackage{amsmath}
\usepackage{wrapfig}
\usepackage{tabularx}
\usepackage{csquotes}
\usepackage{adjustbox}
\usepackage{graphicx}
\usepackage{makecell}
\usepackage{footnote}
\usepackage{lscape}
\usepackage{graphicx}
\graphicspath{{figures/}}
\usepackage{array}
\usepackage{threeparttable}
\usepackage{rotating}
\usepackage[titletoc]{appendix}
\DeclareMathOperator*{\argmax}{arg\,max}
\DeclareMathOperator*{\argmin}{arg\,min}
%\usepackage{afterpage}


\newcommand{\newblanckpage}{
    \newpage
    \thispagestyle{plain} % empty
    \mbox{}
}

\usepackage{mathtools}
\DeclarePairedDelimiter\ceil{\lceil}{\rceil}
\DeclarePairedDelimiter\floor{\lfloor}{\rfloor}

\begin{document}
\newgeometry{margin=1in}
\begin{titlepage}
    \vspace*{1 cm}
    \begin{center}
         {\LARGE \textbf{Raw Afternotes - Deep Learning for NLP}\\}
        \vspace{2 cm}
        \textsc{Ca' Foscari University of Venice}\\
        Department of Environmental Sciences, Informatics and Statistics\\
        \vspace{0.2 cm}
        \begin{figure}[h!]
        	\centering
        	\includegraphics[width=0.6\textwidth]{logo} 
        \end{figure}
        \vspace{0.0 cm} DEEP LEARNING FOR NATURAL LANGUAGE PROCESSING [CM0624]\\
        %\vspace{0.0 cm} Thesis notes\\
        Academic Year 2023 - 2024\\
        \vspace{3.0 cm}
        	
        \begin{flushleft}
        	\begin{tabular}{l l}
        		\textbf{Student} & Zuliani Riccardo 875532\\
                \textbf{Notes} & These notes do not provide any images, they were made to have a logical ladder
        	\end{tabular}
        \end{flushleft}
    \end{center}
\end{titlepage}

\restoregeometry

%\newblanckpage

\pagenumbering{roman}

%\newblanckpage
\tableofcontents

\clearpage

\pagenumbering{arabic}

\setcounter{page}{1}



\part{Lucchese Part}
\chapter{Introduction}
\begin{itemize}
    \item The Moore's Law
    \item Some advantages of multicores:
    \begin{itemize}
        \item Power
        \item Design cost
        \item Defect tollerance
    \end{itemize}
    \item Sequential VS Parallel computing
    \begin{itemize}
        \item Sequential computing
        \item Parallel Computing
    \end{itemize}
    \item Clusters
    \item When is Parallelism Necessary?
\end{itemize}

\chapter{Cache Awareness}
\begin{itemize}
    \item Effect of memory latency
    \item Cache Memory
    \begin{itemize}
        \item Definition
        \item Performance improve in presence of high locality
        \begin{itemize}
            \item Temporal locality
            \item Spartial locality
        \end{itemize}
        \item Cache hit rateo
        \item Other approaches for hiding memory latency
        \begin{itemize}
            \item Multi-threading
            \item Pre-fetching
            \item Drawbacks (Bandwidth and cache pollution)
        \end{itemize}
    \end{itemize}
\end{itemize}
\section{Cache-to-memory Coherence}
\begin{itemize}
    \item After updating/writing data in cache, when to write to memory?
    \item \textbf{Write-Through Policy} definition
    \item \textbf{Write-Back Policy} definition, in case of write data not being in cache
    \begin{itemize}
        \item Write Allocate
        \item Write No Allocate
    \end{itemize}
    \item \textbf{Cache-to-cache Coherence in Symmetric Multi-Processors}
    \begin{itemize}
        \item Snooping Protocols
        \item Cache controller snoop all the bus transactions
        \begin{itemize}
            \item Transaction relevant
            \item Actions to guarantee \textbf{cache coherence}
            \begin{itemize}
                \item \textit{Update}
                \item \textit{Invalidate}
            \end{itemize}
        \end{itemize}
        \item \textbf{Update}
        \begin{itemize}
            \item \textit{Pros:}
            \begin{itemize}
                \item Multiple r/w copies are kept coherent after each write, save bandwidth
            \end{itemize}
            \item \textit{Con:} Unnecessary waste
            \begin{itemize}
                \item Cache block is update but no longer read
                \item Subsequent writes by the same processor cause multiple updates
            \end{itemize}
        \end{itemize}
        \item \textbf{Invalidate}
        \begin{itemize}
            \item \textit{Pro:}
            \begin{itemize}
                \item Multiple writes by the same processor do not cause any additional overhead
            \end{itemize}
            \item \textit{Con:}
            \begin{itemize}
                \item An access that follows an invalidation causes a miss
            \end{itemize}
        \end{itemize}
        \item \textbf{False Sharing}
        \begin{itemize}
            \item Coherent protocols works in term of cache blocks/lines rather single words/bytes
            \item Blocks plays an important role in coherence protocol
            \begin{itemize}
                \item Small blocks the protocols are more efficient
                \item Large blocks are better for spartial locality
            \end{itemize}
        \end{itemize}
    \end{itemize}
    \item \textbf{We want to minimize the cache miss and maximize the cache hit}
    \item External memory model
    \begin{itemize}
        \item Transfer occur in blocks of size \(B\)
        \item The cache has size \(M \geq B\)
        \item With \(M/B\) entries
    \end{itemize}
    \item \textbf{Example of linear scan:} \(\ceil*{\frac{N}{B}+1}\) memory scan
\end{itemize}

\section{Matrix Multiplication}
\begin{itemize}
    \item Rows in row-wise
    \item Columns in column-wise
    \item Each element of matrix \(C\) involves \(O(1+\frac{N}{B})\) transfers
    \item Since there are \(N^2\) elements of \(C\) the complexity is \(O(N^2 \frac{N^3}{B})\)
    \item Approach to reduce the cost
    \item \textbf{Improved Algorithm} complexity \(O\Big(\frac{N^2}{B} + \frac{N^3}{B \cdot \sqrt{M}}\Big)\) with block matrices

    \item \textbf{Cache Oblivious Algorithm}
    \begin{itemize}
        \item Divide-and-conquer approach
        \item We recursively split the matrix until at some point the matrix will fit in cache, whatever is its size
        \item \textit{Recursive Data Layout}, st however we recursively split the matrix, at some point the data will be in consecutive memory location that can be easily loaded in cache
        \item \textbf{Z-Order}
    \end{itemize}
    \item \textbf{Complexity}
    \begin{itemize}
        \item \textit{Case base:} three blocks fit in cache, since successive recursion step do not cause additional misses
        \begin{itemize}
            \item Block size \(k\sqrt{M} \cdot k\sqrt{M}\) for some constant \(k\)
        \end{itemize}
        \item Three of such blocks fill the cache with \(M/B\) misses
        \item Computational complexity of the block-based multiplication is \(\Big(\frac{N}{k\sqrt{M}}\Big)^3\)
        \item Number of misses: \(\Big(\frac{N}{k\sqrt{M}}\Big)^3 \cdot \frac{M}{B} = O\Big(\frac{N^3}{B\sqrt{M}}\Big)\)
        \item When \(N\) is small, data loading: \(O\Big(\frac{N^2}{B}\Big)\)
        \item Total cost: \(O\Big(\frac{N^2}{B} + \frac{N^3}{B \cdot \sqrt{M}}\Big)\)
    \end{itemize}
\end{itemize}


\section{Sorting \& K-Funnel}
\begin{itemize}
    \item Merge sort
    \item \textbf{Cache-aware solution}
    \begin{itemize}
        \item M/B-way merge
        \item \(\theta\Big(\frac{N}{B}\log_{\frac{M}{B}}\frac{N}{M}\Big)\)
    \end{itemize}
    \item \textbf{First Cache Oblivious Solution}
    \begin{itemize}
        \item 2-way merge sort
        \item \(\theta\Big(\frac{N}{B}\log_2\frac{N}{M}\Big)\)
        \item Can we rise the vase of the logarithm to M/M
    \end{itemize}
    \item \textbf{Cache Oblivious Sort: K-Funnel}
    \begin{itemize}
        \item Definition
        \item Description
        \item \item \(\theta\Big(\frac{N}{B}\log_{\frac{M}{B}}\frac{N}{M}\Big)\) like cache-aware solution
    \end{itemize}
\end{itemize}

\section{Multi-core Hierarchies: key challenge}
\begin{itemize}
    \item Good performance for any \(M\) \& \(B\) on 2 levels \textbf{does not} imply \textit{good performance} at all levels of hierarchy
    \item Reason: cache is not fully shared, what is \textbf{good} for \(cpu_1\) is often \textbf{bad} for \(cpu_2\) and \(cpu_3\)
    \item Scheduling of parallel thread has a large impact on cache performance
\end{itemize}

\chapter{Shared Memory: Threads}
\section{Performance Metrics}
\begin{itemize}
    \item two metrics:
    \begin{itemize}
        \item Parallel run time
        \item Cost
    \end{itemize}
    \item \textit{When a parallel algorithm is cost optimal}
    \item \textbf{SpeedUp}
    \begin{itemize}
        \item Formula
        \item Typical success
        \item Linear speedup
        \item Super linear speedup
        \item Using more thread require more overhead
    \end{itemize}
    \item \textbf{Efficiency}
    \begin{itemize}
        \item Measure of resource usage
        \item fraction of time the processors are fully used to execute a fraction 1/p of the best sequential algorithm
        \item Formula
    \end{itemize}
    \item \textbf{Amdahl's Law (SpeedUp)}
    \begin{itemize}
        \item Parallel execution time \(T_p\) cannot be arbitrary reduced by increasing \(p\)
        \item Suppose a fraction \(f\) cannot be executed in parallel
        \item Formula
        \item The sequential fraction \(f\) is an upper bound on the speedup \(S\)
    \end{itemize}
    \item \textbf{Gustafson's Law (Scalability)}
    \begin{itemize}
        \item We can't use a large number of processors in order to process large amount of data
        \item Sequential part c is constant wrt n
        \item \(T(n,p)\) the execution time of the parallelizable part over \(p\) processors, \textit{perfectly parallelizable}
        \item Formula of the scaled speedup
        \item If \(T(n,1)\) increases monotonically with n and the sequential part c is constant, we can achieve linear scaled speedup 
    \end{itemize}
    \item \textbf{Scalability}
    \begin{itemize}
        \item Ability to increase the speedup proportionally to the processors number
        \item Keep efficiency \(E=s/p\) constant when increasing both the processors number and the problem size, such algo are said \textit{scalable}
    \end{itemize}
\end{itemize}

\chapter{OpenMP \& Cache}
\begin{itemize}
    \item Comparison between matrix multiplication algorithm
    \begin{enumerate}
        \item Strategy: \(n^2\) do not fit in cache
        \item Strategy: 
        \begin{itemize}
            \item Pro: cache is better
            \item Cons: 
            \begin{itemize}
                \item Threads are created and destroied many times
                \item Race condition, reading and writing the same memory location
                \item \(n + \#t \cdot n + n\)
            \end{itemize}
        \end{itemize}
        \item Strategy:
        \begin{itemize}
            \item Pro: cache, 1 row an d1 col fit in cache
            \item Con: thread, inner loop takes \(n^2\) times overhead due to thread managment
            \item reduction(+:cij), provate so non race condition
        \end{itemize}
    \end{enumerate}
    \begin{itemize}
        \item First strategy bes one
        \item Second strategy has huge variability since we do not know where the threads will be putted , thus there is the possibility the cache has heavy load already
        \item Third strategy worst one
    \end{itemize}
    \begin{enumerate}
        \setcounter{enumi}{3}
        \item Strategy: Swap the initial two inner loop, reduce the number of creation and elimination of threads
        \item Strategy:
        \begin{itemize}
            \item num\_thread(num\_thread)
            \item nowait, when we finish we move forward the next row
            \item barrier, deciding how many rows are processed each time
            \item best strategy overall
        \end{itemize}
    \end{enumerate}
\end{itemize}

\chapter{Pattern of Paralleism I}
\begin{itemize}
    \item Need to design something quite specific for the problem
    \begin{itemize}
        \item \textbf{Decomposition technique}, generate enough sub-problem to be executed in parallel
        \item \textbf{Mapping}, how is going to executo waht maximizing the load balancing
    \end{itemize}
\end{itemize}

\section{Decomposition}
\begin{itemize}
    \item \textbf{Data-Input Decomposition}
    \item \textbf{Data-Output Decomposition}
    \item \textbf{Exploratory}
    \begin{itemize}
        \item Find the best solution, exploring such space
        \item Exploratory decomposition, solution space partitioned and each partition is explored independently
        \item Only one task will find the solution the others become useless
        \begin{itemize}
            \item Look for a valid solution, tasks are forced to end
            \item Look for the best solution, subtree are associated with an expected of the solution, most promising task are explored first
            \item Task works in parallel, exploration order could be different
        \end{itemize}
    \end{itemize}
\end{itemize}

\subsection{Task Dependency Graph}
\begin{itemize}
    \item Model decomposition and they control dependencies
    \item Not unique for a given problem
    \begin{itemize}
        \item Node = Task
        \item Edges = Task/Data Dependencies
        \item Node Labels = Task Computational Cost
    \end{itemize}
\end{itemize}

\subsection{Parallelism Degree and Critical Path}
\begin{itemize}
    \item \textbf{Parallelism Degree}: number of tasks that can be executed in parallel
    \item \textit{Direct Path in a TDG}: sequence of tasks in the TDG liked by a dependency relation (not in parallel)
    \item \textbf{Critical Path}: longest direct path in the TDG (bottleneck, minimum execution time)
    \item \textit{Average Parallelism Degree}, work/length of the critical path
\end{itemize}

\subsection{Task Interaction Graph}
\begin{itemize}
    \item Data Dependencies task exchange data with others in a decomposition
    \begin{itemize}
        \item Node = Task
        \item Edges = Task/Data Dependencies
        \item Node Labels = Task Computational Cost 
        \item Edges Labels = Amount of data exchanged 
    \end{itemize}
\end{itemize}

\section{Mapping}
\begin{itemize}
    \item \textbf{TDG}: loading balance and minimizing waiting time
    \item \textbf{TIG}: minimizes interaction/communication between tasks
    \item \textbf{Mapping Guidelines}
    \begin{itemize}
        \item Independent task to different processors
        \item Task in Critical Path assigned as early as possible
        \item Minimizing communication cost by scheduling dense subgraphs of the TIG to the same processor
        \item \textbf{Conflict:} tasks to the same processors reduce the communication cost but does not produce any performance improvement
    \end{itemize}
\end{itemize}


\section{Prefix-Sum}
\begin{itemize}
    \item Algorithm
    \item Naive way to parallelize
    \item Analysis of the speedup
    \item \textbf{Exclusive Prefix Sum}
    \begin{itemize}
        \item \textit{UpSweep Pass}
        \begin{itemize}
            \item Complexity
            \item Each node holds the sum of its children
        \end{itemize}
        \item \textit{DownSweep Pass}
        \begin{itemize}
            \item Every Node should have the sum of all the leaves preceding it (before it left most child)
            \item Root = 0
            \item Left child = Parent
            \item Right child = Parent + UpSweep value of its sibling
        \end{itemize}
    \end{itemize}
    \item \(O(\log N)\) Steps and \(O(N)\) summations
    \item Large parallelism degree cna be exploited
\end{itemize}


\chapter{Pattern of Parallelism II - Common Patterns}
\begin{itemize}
    \item \textbf{Pipelines}
    \begin{itemize}
        \item Special kind of task-parallelism
        \item The computation is divided into stages that are executed sequentially
        \item Asymptotically a pipeline achieves a speedup equal to the number of stages
    \end{itemize}
    \item \textbf{Single Program Multiple Data Embarrassingly Parallel}
    \begin{itemize}
        \item Problem that can be solved with a large set of complexity independent sub-tasks (data-parallel)
    \end{itemize}
    \item \textbf{Task Pool}
    \begin{itemize}
        \item \textit{Task List} is stored in a shared data structures
        \item Fixed number of threads
        \item Each one pick a task and execute it, thus synchronization
        \item A thread can generate a new task
    \end{itemize}
    \item \textbf{Dynamic Task Creation}
    \begin{itemize}
        \item Each task can dynamically create new tasks
        \item \textit{Useful for}
        \begin{itemize}
            \item Improve parallelism degree
            \item Split a long task into smaller ones
            \item Adapt to non uniform resources
        \end{itemize}
        \item A task queue has to be shared among a pool of threads, master-worker perform centralized load balancing
        \begin{itemize}
            \item Remove centralization and favour data exchange among neighbors
            \item \textit{Push-Sender Initialized} worker that generates a new task sends it to an other worker
            \item \textit{Pull-Receiver Initialized} when a worker is idle, it asks to other worker for a job to execute (work stealing)
        \end{itemize}
        \item \textit{Partner Selection}
        \begin{itemize}
            \item Random
            \item Global Round Robin (GRR)
            \begin{itemize}
                \item Global variable points to the next worker
                \item A worker that needs a partner reads and increments the global variable
                \item Implement a GRR
            \end{itemize}
            \item Local Round Robin (LRR)
            \begin{itemize}
                \item Every processor keeps a private pointer to the next available worker
                \item No overhead due to sharing a global variable
                \item Approximate a GRR
            \end{itemize}
        \end{itemize}
    \end{itemize}
\end{itemize}

\section{Parallelizing Quicksort}
\subsection{Sequential Version}
\begin{itemize}
    \item Divide-and-conquer, sort a sequence by recursively dividing it into smaller subsequences
    \item \textit{Divide:} a sequence is partitioned into two nonempty subsequences such that each element of the first subsequence is smaller than or equal to each element of the second subsequence
    \item \textit{Conquer:} subsequences are sorted by recursively applying quicksort
    \item This is accomplished by selecting the pivot and using this element to partition the sequence into two parts - one with elements less than or equal to x and the other with elements greater than the pivot
    \item \textbf{Issues:}
    \begin{itemize}
        \item Insufficient parallelism degree
        \item Load imbalance
        \begin{itemize}
            \item Sub-arrays depends on the pivot selection
            \item Their length varies and it is difficult to control
            \item Different length lead to different workloads and imbalance
            \item Wors case, one task get N elements
        \end{itemize}
    \end{itemize}
\end{itemize}

\subsection{Parallel Version}
\begin{itemize}
    \item Assign sub-arrays to processors
    \item Pick a pivot
    \item \textit{Local arrangement} we divide each sub-arrays in a smaller and a greater part respect the pivot
    \item \textit{Global arrangement} merge the the left and right part of each sub-arrays in two bigger sub-arrays
    \item Recursion of the two new sub-arrays until each process can use a sequential algorithm to sort its sub-array
    \item \textbf{Note}
    \begin{itemize}
        \item Higher parallelism degree
        \item Load balance is still sensitive to pivot selection 
    \end{itemize}
\end{itemize}


\section{Odd-Even Transposition}
\begin{itemize}
    \item Based on the bubble sort, which compares every 2 consecutive numbers in the array and swap them if first is greater than the second to get an ascending order array
    \item Repeat n/2 times:
    \begin{itemize}
        \item Compare-exchange odd elements with their immediate neighbour
        \item Compare-exchange even elements with their immediate neighbour
    \end{itemize}
    \item Sequential: \(O(n^2)\)
    \item Parallel cost when \(p = n/2\): \(O(n^2)\)
    \item Complexity of sorting not optimal: \(O(n \log n)\)
    \item Generalized assign batches of \(n/p\) elements to each processor
    \item Local sort on \(n/p\) elements
    \item Algorithm across batches with p phases
    \item Total cost
    \item \(p=O(\log n)\) parallel is cost optimal \(O(n \log n)\)
    \item \textbf{Drawback}: increasing \(p\) need to increase exponentially \(n\) to have scalability
\end{itemize}


\section{Bitonic Sort}
\begin{itemize}
    \item \textbf{Bitonic Sequence}
    \begin{itemize}
        \item \(x_1,...,x_j\) is monotonically increasing
        \item \(x_{j+1},...,x_n\) is monotonically decreasing
    \end{itemize}
    \item \textbf{Bitonic Split}
    \begin{itemize}
        \item Comparator \([i:j]\)
        \item n/2 comparator of the kind \([i: i+n/2]\) for \(\i \leq i \leq n/2\) to a bitonic sequence
        \item proof
    \end{itemize}
    \item \textbf{Bitonic Merge}
    \begin{itemize}
        \item Recursive bitonic split until sequence is sorted in increasing order
        \item \(\log n\) stages, each one perform n/2 compare/swap
        \item complexity and parallel time
    \end{itemize}
    \item \textbf{Bitonic Sort for non-Bitonic Sequence}
    \begin{itemize}
        \item Ascendingly \& Descendingly by recursively invoking bitonic sort
        \item \(\log N\) bitonic merge phases
        \item Sequential complexity
        \item Parallel Complexity (not optimal)
    \end{itemize}
    \item Can be parallelized very easily, it exploit a large parallelism degree
\end{itemize}



\chapter{Large-Scale Data Processing with Map-Reduce}
\begin{itemize}
    \item Everything is build o top key value pairs\
    \begin{itemize}
        \item \(map(k_1, v_1) \rightarrow list(k_2, v_2)\)
        \item \(reduce(k_2, list(v_2), \rightarrow list(k_3, v_3)\)
    \end{itemize}
    \item All in parallel we have a set of mappers and a set of reducers
    \begin{itemize}
        \item \(map(k_1, v_1) \rightarrow list(k_2, v_2)\)
        \item \(shuffle\)
        \item \(reduce(k_2, list(v_2), \rightarrow list(k_3, v_3)\)
    \end{itemize}
    \item \textbf{Coordination: master data structures}
    \begin{itemize}
        \item task status: idle, in-progress, completed
        \item Idle get scheduled as workers become available
        \item When a map task is completed, it send the master the location and sizes of its R intermediate files, one for each reducer
        \item Master pushes this info to reducers
        \item Master ping workers periodically to detect failures
    \end{itemize}
    \item \textbf{Failures}
    \begin{itemize}
        \item \textit{Map Worker Failures}
        \begin{itemize}
            \item Map task completed or in progress are reset to idle
            \item Reduce workers are notified when task is rescheduled on another worker 
        \end{itemize}
        \item \textit{Reduce Worker Failures}
        \begin{itemize}
            \item Only in progress tasks are reset to idle
            \item A different reducer may take the idle task over
        \end{itemize}
        \item \textit{Master Failure}
        \begin{itemize}
            \item Map Reduce task is aborted and client is notified
        \end{itemize}
    \end{itemize}
    \item \textbf{How many Map and Reduce jobs?}
    \begin{itemize}
        \item Make M and R larger than the number of nodes in the cluster
        \item \#Mappers is determined by the \#input split
        \item Many mappers/reducers improve the load balance and speed recovery but increase overhead management
        \item \(R \leq M\)
        \begin{itemize}
            \item System has a maximum capacity of parallel reducers, executed in waves
            \item Shuffling of the first wave is done in parallel by mappers
            \item Shuffling of other waves done later
        \end{itemize}
    \end{itemize}
    \item \textbf{Combiners}
    \begin{itemize}
        \item Map tasks on the same node will produce many pairs with the same key
        \item We can pre-aggregate information after mapper
        \item \(combine(k_2,list(v_2)) \rightarrow list(k_3, v_3)\)
    \end{itemize}
    \item \textbf{Partition Function}
    \begin{itemize}
        \item Inputs are created by continuous split of the input
        \item For reducer, intermediate(k,v) pairs with the same key end up at the same reducer
        \item Hash default partitioning function
    \end{itemize}
    \item \textbf{Is Map Reduce so good?}
    \begin{itemize}
        \item Map reduce is not for performance
        \begin{itemize}
            \item Mapper writes to disk, huge overhead
            \item Shuffle bottleneck
        \end{itemize}
        \item Little coding time
        \begin{itemize}
            \item Override two functions
            \item Not everything can be implemented in terms of map reduce
        \end{itemize}
        \item Fault tolerance
    \end{itemize}
\end{itemize}


\chapter{Vertex Centric Paradigm}
\begin{itemize}
    \item Each node knows about its neighbours (graph structure may change)
    \item Each node may hold some additional custom info
    \item Pros:
    \begin{itemize}
        \item Easy to design
        \item Large parallelism
        \item Can be implemented over map reduce
    \end{itemize}
    \item Con: local view may lead to sub-optimal results
\end{itemize}

\section{Label Propagation}
\begin{itemize}
    \item \((X, C_{min})\), meaning that \(X\) belongs to the connected component with id \(C_{min}\), initially \(X\) belongs only to itself
    \item Each node knows its neighbours
    \item Iterative algorithm
    \item Algorithm
    \item \(O(d)\) with d is the diameter
\end{itemize}

\section{Hash-To-Min}
\begin{itemize}
    \item List of \((X, C)\), meaning that \(X\) knows about nodes in the set \(C\), initially \(X\) knows C=X plus its neighbours
    \item Algorithm
    \item \(O(\log d)\) with d is the diameter
\end{itemize}











\part{Bruch Part}
\chapter{Ranking}
\begin{itemize}
    \item Ranking Dataset
    \begin{itemize}
        \item Explicit Feedback: human assessors are presented with a query and a ranked list, and are asked to grade each document with respect to the query, challenge
        \item Implicit Feedback: as user interact with a ranking system, collect signals that are indicative of relevance, challenge
    \end{itemize}

    \item Ranking Metrics
        \begin{itemize}
        \item \textbf{Ranking as a classification or regression}, problem: we do not care on label value, we care only if a document is relevant or not
        \item \textbf{Rank correlation (kendall's \(\tau\))}, pairwise classification if the documents are correctly ordered, problem: the bottom and the upper part are equally weighted 
        \item \textbf{Reciprocal rank at K}, position of the first relevant document over the top k position
        \item \textbf{Precision at rank K}, fraction of documents in the top -k set having \(y_i > 0\)
        \item \textbf{Average Precision at rank K}, gain (tells us if it is a good document) and discount (intuitively it is the probability of been visited)
    \end{itemize}

    \item Learning to Rank / Ml framework
    \begin{itemize}
        \item Input space
        \item Output space
        \item Hypothesis space
        \item Loss function
    \end{itemize}

    \item Learning to Rank proprieties
    \begin{itemize}
        \item Feature based
        \item Discriminative training
    \end{itemize}
\end{itemize}

\chapter{Ranking Loss Function}
\begin{itemize}
    \item \textbf{Point Wise Losses}
    \begin{enumerate}
        \item Input space: feature vector of each single document
        \item Output space: relevance of each single document
        \item Hypothesis space: feature vector as input and predict the relevance degree (document)
        \item Loss function: measure the accurate prediction of the ground truth
    \end{enumerate}
    \begin{itemize}
        \item Ranking as ordinal regression
        \item Ranking as binary classification, sigmoid
    \end{itemize}
    \item \textbf{Pair Wise Losses}
    \begin{enumerate}
        \item Input space: pair of documents as feature vectors
        \item Output space: pairwise preferences between each doc pair
        \item Hypothesis space: bivariate functions, pair of docs as input and output the relative order
        \item Loss function: measure the inconsistency between the obtained order and the ground truth
    \end{enumerate}
    \begin{itemize}
        \item Ranking as preference learning: ranknet, ranking svm
        \item From ranknet to lambda-rank
    \end{itemize}
    \item \textbf{List Wise Losses}
    \begin{enumerate}
        \item Input space: group of documents associated witht he query
        \item Output space: relevance degree of all the documents associated to the query, ranked list of all the documents
        \item Hypothesis space: Multivariate functions \(h\), operate on a group of docs and predict their relevance
        \item Loss function: can we use greadient descent?
    \end{enumerate}
    \begin{itemize}
        \item Listnet
    \end{itemize}
    \item \textbf{NDCG Consistency}
\end{itemize}

\chapter{Representation and Hypothesis Classes}
\begin{itemize}
    \item \textbf{Gradient Boosting Decision Trees}
    \begin{itemize}
        \item Weak learner
        \item We can not use gradient descend in a linear function, not differentiable
        \item So regression, with loss MSE
    \end{itemize}
    \item \textbf{Neural Network and Pre-Trained Transformer}
    \begin{itemize}
        \item Representing word as vectors:
        \begin{itemize}
            \item Skip-grap
            \item Continuous Bag of words
        \end{itemize}
        \item Learning word embedding: word2vec
        \item \textit{Context Embedding}
        \begin{itemize}
            \item Bert (Bidirectional Encoder Representation for Transformers)
            \begin{itemize}
                \item Tokenization
                \item Input Encoding, wordpiece
                \item Model Architecture, deep bidirectional transformer: self-attention and feed-forward
                \item Masked Language Model, predict missing word by random masking
                \item Next Sentence Prediction: sentence consecutively or not
                \item Pre-train and fine tuning
            \end{itemize}
            \item MonoBert
            \begin{itemize}
                \item Training identical as the Bert
                \item  Data fro training and fine tuning are specific tot he target language
            \end{itemize}
            \item DuoBert
            \begin{itemize}
                \item Multilingual corpus
                \item Language identification
                \item Tokenization and input encoding
                \item Multilingual model architecture
                \item Multilingual MLM and NSP
                \item Pretrained and fine tuning
            \end{itemize}
        \end{itemize}
    \end{itemize}
\end{itemize}

\chapter{Complexity of Ranking Functions}
\begin{itemize}
    \item \textbf{Knowledge Distillation:} given a large model find a smaller more efficient model that is just as effective
    \begin{itemize}
        \item Pruning nodes in trees
        \item Discarding Redundant Tree in Forest: n trees, impact score, discard tree with lowest impact score, repeat until reach p\% is removed
        \item Learning a NN from a tree: lambda-mart
    \end{itemize}
    \item \textbf{Early Exit Algorithm:} perform approximate inference with partial evaluation instead
    \begin{itemize}
        \item Placing exit points in decision tree: decide to discard a doc at an exit point given a score/rank threshold
        \item Placing exit point in layed NN: cascade transformer
    \end{itemize}
    \item \textbf{Ranking cascade:} apply mode complex models to progressively smaller set of documents
    \begin{itemize}
        \item Less sophisticated ranking functions are cheap but less accurate
        \item More sophisticated ranking functions are expensive but more accurate
    \end{itemize}
\end{itemize}


\chapter{Retrieval with MIPS: Representation Learning}
\section{Sparse Representation}
\begin{itemize}
    \item \textbf{Term Matching}
    \begin{itemize}
        \item TF-IDF
        \item BM25
        \item TF-IDF and BM25 ans inner product of vectors
        \item Learning term importance
        \item Learning term Frequencies
        \item Vocabulary miss match problem
        \item \textbf{SPLADE (Supervised Progressive Learning for Approximate Dictionary based Entity extraction)}
        \begin{itemize}
            \item Seed terms: start with an initial seed 
            \item Progressive learning: expand the dictionary terms
            \item Contextual extraction: it consider the contextual info to improve extraction accuracy
            \item Supervised learning: classifier for entity extraction
            \item Approximate matching: account for variations in entity names by allowing approximate matching
        \end{itemize}
        \item \textbf{Document Expansion}
        \begin{itemize}
            \item Anticipating queries from text
            \item \textbf{Doc2Query}: automatically generates queries that are representative of the information contained in the doc
            \begin{itemize}
                \item Pre-trained with LM: bert
                \item Fine-tune with D-Q pairs: doc and human generated query, helps the model learn to generate queries that are relevant given docs
                \item Masked language model objective: mask word in both doc and query and train the model to predict those
                \item Joint D-Q encoding: capture the relationship between the model and the desired query representation
                \item Generation and ranking: multiple candidates queries and ranks according to the doc
            \end{itemize}
        \end{itemize}
    \end{itemize}
\end{itemize}

\section{Dense Representation}
\begin{itemize}
    \item Cross-Encoders vs Bi-Encoders
    \item Representing words as vectors
    \begin{itemize}
        \item Skip-gram
        \item Continuous bag of words
    \end{itemize}
    \item \textbf{Sentence BERT}
    \begin{itemize}
        \item Pretrained Transformer model: BERT
        \item Siamese of triplet network architecture
        \item Contrastive objective for fine tuning
        \item Sentence Embedding
        \item Semantic sentence similarity
    \end{itemize}
    \item Deep Passage Retrieval
    \item How to find non relevant document: ANCE
\end{itemize}

\chapter{Retrieval with MIPS: Sparse Vectors}
\begin{itemize}
    \item \textbf{The Inverted Index}
    \begin{itemize}
        \item TAAT-Retrieval
        \item DAAT-Retrieval
    \end{itemize}
    \item Dynamic Pruning for TAAT-Early termination
    \begin{itemize}
        \item Upper Bound
        \item Threshold
    \end{itemize}
    \item Dynamic Pruning for DAAT
    \begin{itemize}
        \item Assumption: non negativity and zipfian's distribution of items
        \item Concepts: upperbound and threshold
        \item \textit{Max Score}
        \item \textit{WAND}
    \end{itemize}
\end{itemize}

\chapter{Retrieval with MIPS: Dense Vectors}
\begin{itemize}
    \item Quantization
    \begin{itemize}
        \item Distance approximation
        \item Query execution
    \end{itemize}
    \item Product Quantization
    \begin{itemize}
        \item Subspace quantization
        \item Query execution
    \end{itemize}
    \item IVFPQ
    \item Clustering methods - Two Step Approximate MIPS
    \item Graph metods
    \begin{itemize}
        \item Voronoi Regions
        \item Delaunay Graph
        \item Approximate the Delaunay Graph 
    \end{itemize}
\end{itemize}

\end{document}

