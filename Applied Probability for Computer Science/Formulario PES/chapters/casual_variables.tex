\chapter{Variabili Casuali}
\begin{tcolorbox}[title=Variabili Casuali]
Una \textbf{variabile aleatoria} o \textbf{casuale} \(X\) è una funzione che assume valori numerici determinati dall’esito di un certo fenomeno aleatorio. Formalmente, se \(\Omega\) è lo spazio campionario relativo al fenomeno di interesse, \(X\) è una particolare funzionw
\[X : \Omega \rightarrow \mathbb R\]
\end{tcolorbox}

\begin{tcolorbox}[title=Funzione di Ripartizione]
Si dice \textbf{funzione di ripartizione} (o di distribuzione cumulativa) di una variabile aleatoria \(X\) la funzione \(F : \mathbb{R} \rightarrow [0, 1]\) così definita:
\[F(x) = \mathbb{P}[P \leq x], \quad \forall i \in \mathbb{R}\]

La funzione di ripartizione \(F\) ha le seguenti proprietà:
\begin{itemize}
    \item \(F\) è non decrescente
    \item \(F\) è continua a destra
    \item \(\lim_{x \rightarrow - \infty}F(x)=0\) e \(\lim_{x \rightarrow + \infty}F(x)=1\)
\end{itemize}
\end{tcolorbox}

\subsection{Proprietà del Valore Atteso}
\begin{itemize}
    \item \(\mathbb{E}[a] = a\), dove \(a\) è una costante
    \item \(\mathbb{E}[aX + b] = a\mathbb{E}[X] + b\), dove \(a\) e \(b\) sono costanti
\end{itemize}

\subsection{Proprietà della Varianza}
\begin{itemize}
    \item \(Var[a] = 0\), dove \(a\) è una costante
    \item \(Var[aX + b] = a^2Var[X]\), dove \(a\) e \(b\) sono costanti
\end{itemize}

\section{Variabili Aleatorie Discrete}
\begin{tcolorbox}[title=Variabile Aleatoria Discreta]
Una \textbf{varaibile aleatoria discreta} \(X\) assume valori in un insieme numerabile (o finito) di punti, \({x_1,x_2,...,x_i,...})\).

Proprieta':
\begin{itemize}
    \item \(0 \leq p_i \leq 1,    \quad    \forall i = 1,2,...\)
    \item \(\sum_i p_i = 1\)
    \item \(\mathbb{P}[X \in A] = \sum_{i:x_i \in A}p_i\)
\end{itemize}
\end{tcolorbox}

\subsubsection{Funzione di Probabilità}
Un’assegnazione di probabilità per \(X\), \(P(x) = \mathbb{P}[X = x]\) viene chiamata funzione di probabilità e può essere rappresentata graficamente tramite un diagramma a bastoncini

\subsubsection{Funzione di Ripartizione}
\[F(X) = \sum_{i:x_i\leq x}\mathbb{P}[X = x_i] = \sum_{i:x_i \leq x}p_i\]

\subsubsection{Valore Atteso \(\mathbb{E}[X]\)}
\[\mathbb{E}[X] = \sum_i x_ip_i = x_1p_1 + x_2p_2 + ...\]

\subsubsection{Varianza}
\[Var[X] = \sum_i (x_i - \mathbb{E}[X])^2p_i \rightarrow \sum_ix_i^2p_i - [\mathbb{E}[X]]^2\]

\section{Variabili Aleatorie Continue}
\begin{tcolorbox}[title=Variabile aleatoria continua]
Una \textbf{variabile aleatoria continua} \(X\) assume valori in un insieme continuo di punti (un sottoinsieme di \(\mathbb{R}\) non numerabile).

La curva è il grafico di una funzione \(f(x)\) tale che::
\begin{itemize}
    \item \(f(x) \geq 0, \quad \forall x\in \mathbb{R}\)
    \item \(\int_{\mathbb{R}} f(x)\,dx = 1\), cioè l'area totale sotto il grafico di \(f(x)\) è 1.
\end{itemize}
\end{tcolorbox}

\newpage
\subsubsection{Densità di Probabilità}
Una funzione \(f(x)\) con le proprietà precedenti viene chiamata densità di probabilità.

Una volta assegnata una densità di probabilità alla variabile \(X\), si può scrivere, per ogni evento \(A\) di \(R\): 
\[\mathbb{P}[X \in A] = \int_{A}f(x)\,dx\]

\subsubsection{Funzione di Ripartizione}
\[F(x) = \int_{-\infty}^x f(t)\,dt\]

\subsubsection{Valore Atteso \(\mathbb{E}[X]\)}
\[\mathbb{E}[X] = \int_{\mathbb{R}}xf(x)\,dx\]

\subsubsection{Varianza}
\[Var[X] = \int_{\mathbb{R}}(x - \mathbb{E}[X])^2f(x)\,dx \rightarrow \int_{\mathbb{R}}x^2f(x)\,dx - [\mathbb{E}[X]]^2\]

\section{Moda, Mediana e Quantili}
\subsubsection{Moda}
La \textbf{moda} di una va \(X\) è il punto (o i punti) in cui la funzione di probabilità (o di densità) assume il valore massimo.

\subsubsection{Mediana}
La \textbf{Mediana} di una va \(X\) è il minino valore \(m\) per cui
\[F(m) = \mathbb{P}[X \leq m] \geq \frac{1}{2}\]

\subsubsection{Quantili}
Fissato un valore \(\alpha \in (0,1)\), il \textbf{quantili di livello \(\alpha\)} di una va \(x\) è il minimo valore \(q_{\alpha})\) per cui
\[F(q_{\alpha}) = \mathbb{P}[X \leq q_{\alpha}] \geq \alpha\]