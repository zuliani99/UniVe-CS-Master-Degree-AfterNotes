\chapter{Consistency}
\begin{itemize}
    \item \textbf{Consistency:}
    \begin{itemize}
        \item When a server modifies a value in one replica, the value must be immediately modified in all replicas
        \item Costly for great number of replicas
        \item Might block a read access until all replicas are updated 
    \end{itemize}
    \item \textbf{Availability:}
    \begin{itemize}
        \item High performance of query answering
        \item Very low response time
    \end{itemize}
    \item \textbf{Partition Tolerance:}
    \begin{itemize}
        \item Failures in distributed systems do not lead to failure of the entire system
    \end{itemize}
\end{itemize}

\begin{tcolorbox}[title=\textbf{CAP conjecture}]
\textbf{C}onsistency, \textbf{A}vailability, \textbf{P}artition Tolerance: \textit{Pick any two!}
\end{tcolorbox}

\section{Strong Consistency}
After update, any subsequent access returns the updated value. However we can observe \textbf{stale read} and \textbf{lost update}.

\section{Weak Consistency}
\begin{itemize}
    \item Ensuring strong consistency is usually costly
    \item Strong consistency requires a high amount of synchronization between replicas
    \item \textbf{Weak consistency} can improve performance of the overall system
    \item However, weak consistency can cause conflicts and inconsistencies of the data and may even lead to data loss
\end{itemize}

\subsubsection{Inconsistency window}
The duration from the acceptance of a write request by the first replica until the last successful write execution by all replicas is called the \textbf{inconsistency window}. Ideally, the inconsistency window is only very short in order to reduce the amount of stale reads.

\subsubsection{Eventual consistency}
\begin{itemize}
    \item Replicas may hold divergent versions of a data item due to concurrent updates and propagation delays
    \item Replicas agree on a version after some time of inconsistency and as long as no new updates are issued by users
    \item Eventual consistency requires the two properties:
    \begin{enumerate}
        \item Total propagation of writes to all replicas
        \item Convergence of all replicas towards a unique common value
    \end{enumerate}
\end{itemize}

\section{Write and Read Quorums}
A flexible mechanism to avoid stale reads and lost updates among a group of servers in a replicated database system is to use quorums when reading and writing data:
\begin{itemize}
    \item \textit{Read quorum} is a subset of replicas that have to be contacted when \textbf{reading data}
    \begin{itemize}
        \item For a \textbf{successful read}, all replicas in a read quorum have to return the \textit{same answer value}
    \end{itemize}
    \item \textit{Write quorum} is a subset of replicas that have to be contacted when \textbf{writing data}
    \begin{itemize}
        \item For a \textbf{successful write}, all replicas in the write quorum have to \textit{acknowledge the write request}
    \end{itemize}
\end{itemize}
A usual requirement for a quorum-based system is that any read and write \textbf{quorums overlap:} \textbf{the sum of read quorum size and write quorum size are larger than the replication factor}. In this way, it can be ensured that at least one replica has acknowledged all previous writes and hence is able to return the most recent value.

Quorum rules for $N$ \textbf{replicas}, \textbf{read} quorum size $R$, \textbf{write} quorum size $W$:
\begin{itemize}
    \item $R + W > N$ for \textit{consistent read}
    \item $2W > N$ for \textit{consistent write}
\end{itemize}

We typically have two typical variants of quorum-based systems:
\begin{itemize}
    \item \textbf{Read-one write-all (ROWA):} writes are acknowledged by all replicas, but for reads it suffices to contact one replica
    \item \textbf{Majority Quorum:} or both reads and writes requires that (for $N$ replicas) at least $\frac{N}{2} + 1$ replicas acknowledge the writes and at least $\frac{N}{2} + 1$ replicas are asked when reading a value
\end{itemize}
When all replicas in a read quorum return an identical value, the requesting client can be sure that this is the most recent value and hence the read value is strongly consistent.

We mention also that \textbf{Partial quorums} only ensure weak consistency (for example, during failures) or when strong consistency is not
required.