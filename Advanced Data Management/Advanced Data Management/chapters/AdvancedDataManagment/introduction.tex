\chapter{Introduction}
\section{Database Proprieties}
Database systems should guarantee a correct and reliable execution
in several use cases. Database system fulfill the following properties:
\begin{itemize}
    \item \textbf{Data management:} 
    \begin{itemize}
        \item A database system not only stores data, it also support operations for retrieval of data, search and updates. 
        \item Database system should also support transactions: a sequence of operations on data in a database that must not be interrupted, or in others words either all operations succeed to their full extent or none of the operations is executed.
    \end{itemize}
    \item \textbf{Scalability:} processing data by its distribution in a network of database servers and a high level of parallelization.
    \item \textbf{Heterogeneity:} when collecting data or producing data, could be saved in different type of structure:
    \begin{itemize}
        \item \textit{Structured:} if the data is in relational format which describe a fixed schema
        \item \textit{Semi-structured:} like tree or more in general graphs
        \item \textit{Unstructuired:} text documents
    \end{itemize}
    \item \textbf{Efficiency:} the majority of database applications need fast database systems, so fast storage and retrieval.
    \item \textbf{Persistence:} providing a long-term storage facility for data, implementing also recovery system like transaction manager to allow to go back after a crash or continue if the operation has finish correctly. (Transaction could take some time)
    \item \textbf{Reliability:} database must prevent data loss and they have to support some kind of recovery system, thus they have to ensure:
    \begin{itemize}
        \item \textit{Data integrity:} data stored in the database system should not be distorted unintentionally
        \item \textit{Physical redundancy:} storing copies of data on other servers
    \end{itemize}
    \item \textbf{Consistency:} 
    \begin{itemize}
        \item Database system must do its best to ensure that no incorrect or contradictory data persist in the system.
        \item Automatic verification of consistency constraints
        \item Automatic update of distributed data copies
    \end{itemize}
    \item \textbf{Non-redundancy:} 
    \begin{itemize}
        \item Duplication of values inside the stored data sets should be avoided
        \item Data sets with logical redundancy are prone to different forms of anomalies
        \item \textit{Normalization} is one way to transform data sets into a non-redundant format
    \end{itemize}
    \item \textbf{Multi-User Support:} support concurrent accesses by multiple users or applications, with the constraint of isolation and not interference between users. A major issue is the control access: in which data must be protected like the usage of \textit{views} or implementing an \textit{authentication mechanism}
\end{itemize}

\section{DB Design}
The design phase should clearly answer basic questions like:
\begin{itemize}
    \item Which data are relevant for the customers or the external applications?
    \item How should these relevant data be stored in the database? 
    \item Which are usual access patterns on the stored data?
\end{itemize}
For conventional database systems changing the schema on a running database system is complex and costly, thus database design should be done with due care and following design criteria like the following:
\begin{itemize}
    \item \textbf{Completeness:} all aspects of the information needed by the accessing applications should be covered
    \item \textbf{Soundness:} all information aspects and relationships between different aspects should be modeled correctly
    \item \textbf{Minimality:} no unnecessary or logically redundant information should be stored
    \item \textbf{Readability:} no complex encoding should be used
    \item \textbf{Modifiability:} changes in the structure of the stored data are likely to occur when running a database system over a long time
    \item \textbf{Modularity:} the entire data set should be divided into subsets
\end{itemize}
There are several graphical languages for database design, like Entity-Relationship Model (ERM) and the Unified Modeling Language (UML). For further information refer to the book.