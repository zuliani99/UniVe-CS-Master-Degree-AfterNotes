\chapter{Access Methods Management}
The \textit{Access Methods Manager} provides to the \textit{Relational Engine} the operators used by its modules to execute the commands for the definition and use of databases.

\section{The Storage Engine}
An abstract database machine is normally divided into two parts:
\begin{itemize}
    \item An abstract machine for the logical data model, \textbf{Relational Engine}
    \item An abstract machine for the physical data model, \textbf{Storage Engine}
\end{itemize}
The Relational Engine includes modules to support the execution of SQL commands, by interacting with the Storage Engine.

While the interface of the relational engine depends on the data model features, the interface of the storage engine depends on the data structures used in permanent memory.

The operators on data exported by the storage engine are procedural and can be grouped into the following categories:
\begin{itemize}
    \item Operators to \textit{create} databases
    \item Operators to \textit{start} and to \textit{end} a transaction
    \begin{itemize}
        \item \textit{beginTransaction}
        \item \textit{commit}
        \item \textit{abort}
    \end{itemize}
    \item Operators on \textit{heap files} and \textit{indexes}
    \item Operators about \textit{access methods} available for each relation
\end{itemize}

\section{Access Method Operators}
\begin{tcolorbox}
The Access Methods Manager provides the operators to transfer data between permanent memory and main memory in order to answer a query on a database.
\end{tcolorbox}

Permanent data are organized as collections of records, stored in \textbf{heap files}, and indexes are optional auxiliary data structures associated with a collection of records.

An \textbf{index} consists of a collection of records of the form \textit{(key value RID)}, where
\begin{itemize}
    \item \textit{key value} is a value for the search key of the index
    \item \textit{RID} is the identifier of a record in the relation being indexed
\end{itemize}
The operators provided by the Access Methods Manager are used to implement the operators of physical query plans generated by the query optimizer.

The records of a heap file can be retrieved by a serial scan or directly using their RID obtained by an index scan with a search condition.

A heap file or index scan operation is implemented as an iterator, also called cursor, which is an object with methods that allow a consumer of the operation to get the result one record at a time.

When a heap file iterator is created, it is possible to specify the RID of the first record to return. Instead, when an index iterator is created a key range is specified.