\chapter{MongoDB}
\section{Document}
MongoDB stores data records as BSON documents. BSON is a binary representation of JSON documents, though it contains more data types than JSON. 

\subsection{Document Structure}
The value of a field can be any of the BSON data types, including other documents, arrays, and arrays of documents. Field names are strings. Documents have the following restrictions on field names:
\begin{itemize}
    \item The field name \(\_id\) is reserved for use as a primary key; its value must be unique in the collection, is immutable, and may be of any type other than an array. If the \(\_id\) contains subfields, the subfield names cannot begin with a (\$) symbol.
    \item Field names cannot contain the null character.
    \item The server permits storage of field names that contain dots (.) and dollar signs (\$)
    \item 
\end{itemize}

\subsection{Dot Notation}
MongoDB uses the dot notation to access the elements of an array and to access the fields of an embedded document.
To specify or access a field of an embedded document with dot notation, concatenate the embedded document name with the dot (.) and the field name, and enclose in quotes

\subsection{Document Limitations}
Documents have the following attributes:
\begin{itemize}
    \item \textbf{Document Size Limit:} The maximum BSON document size is 16 megabytes.
    \item \textbf{Document Field Order:} Unlike JavaScript objects, the fields in a BSON document are ordered.
    \item \textbf{The \(\_id\) Field:} In MongoDB, each document stored in a collection requires a unique \_id field that acts as a primary key. If an inserted document omits the \_id field, the MongoDB driver automatically generates an ObjectId for the \_id field.
\end{itemize}

\subsection{Other Uses of the Document Structure}
\begin{itemize}
    \item \textbf{Query Filter Documents:} Query filter documents specify the conditions that determine which records to select for read, update, and delete operations. You can use \(<field>:<value>\) expressions to specify the equality condition and query operator expressions
    \item \textbf{Update Specification Documents:} Update specification documents use update operators to specify the data modifications to perform on specific fields during an update operation.
    \item \textbf{Index Specification Documents:} Index specification documents define the field to index and the index type
\end{itemize}

\section{Databases and Collections}
MongoDB stores data records as documents (specifically BSON documents) which are gathered together in collections. A database stores one or more collections of documents.

\subsection{Database}
In MongoDB, databases hold one or more collections of documents. If a database does not exist, MongoDB creates the database when you first store data for that database.

\subsection{Collections}
MongoDB stores documents in collections. Collections are analogous to tables in relational databases.
\begin{itemize}
    \item \textbf{Create a Collection:} If a collection does not exist, MongoDB creates the collection when you first store data for that collection.
    \item \textbf{Explicit Creation:} MongoDB provides the db.createCollection() method to explicitly create a collection with various options, such as setting the maximum size or the documentation validation rules.
    \item \textbf{Document Validation:} By default, a collection does not require its documents to have the same schema; i.e. the documents in a single collection do not need to have the same set of fields and the data type for a field can differ across documents within a collection.
    \item \textbf{Modifying Document Structure:} To change the structure of the documents in a collection, such as add new fields, remove existing fields, or change the field values to a new type, update the documents to the new structure.
    \item \textbf{Unique Identifiers:} Collections are assigned an immutable UUID. The collection UUID remains the same across all members of a replica set and shards in a sharded cluster.
\end{itemize}

\section{MongoDB Query API}
A document in MongoDB is a data structure composed of field and value pairs. Documents are stored as BSON which is the binary representation of JSON. This low level of abstraction helps you develop quicker and reduces the efforts around querying and data modeling. The document model provides several advantages, including:
\begin{itemize}
    \item Documents correspond to native data types in many programming languages.
    \item Embedded documents and arrays reduce need for expensive joins.
    \item Flexible schema. Documents do not need to have the same set of fields and the data type for a field can differ across documents within a collection.
\end{itemize}

\section{MongoDB CRUD Operations}
https://www.mongodb.com/docs/manual/crud/

\section{Aggregation Pipeline}
An aggregation pipeline consists of one or more stages that process documents:
\begin{itemize}
    \item Each stage performs an operation on the input documents. For example, a stage can filter documents, group documents, and calculate values.
    \item The documents that are output from a stage are passed to the next stage.
    \item An aggregation pipeline can return results for groups of documents. For example, return the total, average, maximum, and minimum values.
    \item The same stage can appear multiple times in the pipeline with these stage exceptions: \(\$out\), \(\$merge\), and \(\$geoNear\)
    \item To calculate averages and perform other calculations in a stage, use aggregation expressions that specify aggregation operators. You will learn more about aggregation expressions in the next section.
\end{itemize}

\subsection{Aggregation Pipeline Expressions}
Some aggregation pipeline stages accept an aggregation expression, which:
\begin{itemize}
    \item Specifies the transformation to apply to the current stage's input documents.
    \item Transform the documents in memory.
    \item Can specify aggregation expression operators to calculate values.
    \item Can contain additional nested aggregation expressions.
\end{itemize}

\section{Replication}
A replica set in MongoDB is a group of mongod processes that maintain the same data set. Replica sets provide redundancy and high availability, and are the basis for all production deployments.

\subsection{Redundancy and Data Availability}
Replication provides redundancy and increases data availability. With multiple copies of data on different database servers, replication provides a level of fault tolerance against the loss of a single database server.

In some cases, replication can provide increased read capacity as clients can send read operations to different servers. Maintaining copies of data in different data centers can increase data locality and availability for distributed applications. You can also maintain additional copies for dedicated purposes, such as disaster recovery, reporting, or backup.

\subsection{Replication in MongoDB}
\begin{itemize}
    \item A replica set is a group of mongod instances that maintain the same data set.
    \item  A replica set contains several data bearing nodes and optionally one arbiter node.
    \item Of the data bearing nodes, one and only one member is deemed the primary node, while the other nodes are deemed secondary nodes.
    \begin{itemize}
        \item The \textbf{primary node} receives all write operations.
        \item The \textbf{secondaries} replicate the primary's oplog and apply the operations to their data sets such that the secondaries' data sets reflect the primary's data set. If the primary is unavailable, an eligible secondary will hold an election to elect itself the new primary. 
    \end{itemize}
\end{itemize}

\subsection{Asynchronous Replication}
Secondaries replicate the primary's oplog and apply the operations to their data sets asynchronously. By having the secondaries' data sets reflect the primary's data set, the replica set can continue to function despite the failure of one or more members.

\subsubsection{Replication Lag and Flow Control}
Replication lag refers to the amount of time that it takes to copy (i.e. replicate) a write operation on the primary to a secondary. Some small delay period may be acceptable, but significant problems emerge as replication lag grows, including building cache pressure on the primary.

\subsection{Automatic Failover}
When a primary does not communicate with the other members of the set for more than the configured electionTimeoutMillis period, an eligible secondary calls for an election to nominate itself as the new primary. The cluster attempts to complete the election of a new primary and resume normal operations.

The replica set cannot process write operations until the election completes successfully. The replica set can continue to serve read queries if such queries are configured to run on secondaries while the primary is offline.

Lowering the electionTimeoutMillis replication configuration option from the default 10000 (10 seconds) can result in faster detection of primary failure. However, the cluster may call elections more frequently due to factors such as temporary network latency even if the primary is otherwise healthy.

\section{Read Operations}
By default, clients read from the primary [1]; however, clients can specify a read preference to send read operations to secondaries.
\begin{itemize}
    \item \textbf{Asynchronous replication:} to secondaries means that reads from secondaries may return data that does not reflect the state of the data on the primary.
    \item \textbf{Multi-document transactions:} hat contain read operations must use read preference primary. All operations in a given transaction must route to the same member.
\end{itemize}

\subsection{Data Visibility}
\begin{itemize}
    \item Depending on the read concern, clients can see the results of writes before the writes are durable.
    \item For operations in a multi-document transaction, when a transaction commits, all data changes made in the transaction are saved and visible outside the transaction. That is, a transaction will not commit some of its changes while rolling back others.
    \item Until a transaction commits, the data changes made in the transaction are not visible outside the transaction.
\end{itemize}

\subsection{Mirrored Reads}
Mirrored reads reduce the impact of primary elections following an outage or planned maintenance. After a failover in a replica set, the secondary that takes over as the new primary updates its cache as new queries come in.

\subsection{Transactions}
Multi-document transactions that contain read operations must use read preference primary. All operations in a given transaction must route to the same member.

\section{Sharding}
Sharding is a method for distributing data across multiple machines. MongoDB uses sharding to support deployments with very large data sets and high throughput operations. There are two methods for addressing system growth: vertical and horizontal scaling.
\begin{itemize}
    \item \textbf{Vertical Scaling:} involves increasing the capacity of a single server, such as using a more powerful CPU, adding more RAM, or increasing the amount of storage space.
    \item \textbf{Horizontal Scaling:} involves dividing the system dataset and load over multiple servers, adding additional servers to increase capacity as required.
    \begin{itemize}
        \item Better efficiency than a single high-speed high-capacity server
        \item Expanding the capacity of the deployment only requires adding additional servers, so lower cost
    \end{itemize}
\end{itemize}

\subsection{Sharded Cluster}
A MongoDB sharded cluster consists of the following components:
\begin{itemize}
    \item \textbf{shard:} Each shard contains a subset of the sharded data. Each shard can be deployed as a replica set.
    \item \textbf{mongos:} The mongos acts as a query router, providing an interface between client applications and the sharded cluster.
    \item \textbf{config servers:} Config servers store metadata and configuration settings for the cluster.
\end{itemize}

\subsection{Shard Keys}
MongoDB uses the shard key to distribute the collection's documents across shards. The shard key consists of a field or multiple fields in the documents.
\begin{itemize}
    \item You select the shard key when sharding a collection.
    \item A document's shard key value determines its distribution across the shards.
\end{itemize}
\subsubsection{Shard Key Index}
To shard a populated collection, the collection must have an index that starts with the shard key.

\subsubsection{Shard Key Strategy}
The choice of shard key affects the performance, efficiency, and scalability of a sharded cluster. A cluster with the best possible hardware and infrastructure can be bottlenecked by the choice of shard key.

\subsection{Advantages of Sharding}
\begin{itemize}
    \item \textbf{Reads / Writes}: MongoDB distributes the read and write workload across the shards in the sharded cluster, allowing each shard to process a subset of cluster operations.
    \item \textbf{Storage Capacity:} Sharding distributes data across the shards in the cluster, allowing each shard to contain a subset of the total cluster data. 
    \item \textbf{High Availability:} The deployment of config servers and shards as replica sets provide increased availability.
\end{itemize}

\subsection{Considerations Before Sharding}
\begin{itemize}
    \item Sharded cluster infrastructure requirements and complexity require careful planning, execution, and maintenance.
    \item Once a collection has been sharded, MongoDB provides no method to unshard a sharded collection.
    \item While you can reshard your collection later, it is important to carefully consider your shard key choice to avoid scalability and perfomance issues.
\end{itemize}

\subsection{Sharded and Non-Sharded Collections}
A database can have a mixture of sharded and unsharded collections. Sharded collections are partitioned and distributed across the shards in the cluster. Unsharded collections are stored on a primary shard. Each database has its own primary shard.

\subsection{Connecting to a Sharded Cluster}
You must connect to a mongos router to interact with any collection in the sharded cluster. This includes sharded and unsharded collections.

\subsection{Sharding Strategy}
\begin{itemize}
    \item \textbf{Hashed Sharding:} involves computing a hash of the shard key field's value. Each chunk is then assigned a range based on the hashed shard key values.
    \item \textbf{Ranged Sharding:} involves dividing data into ranges based on the shard key values. Each chunk is then assigned a range based on the shard key values.
\end{itemize}

\subsection{Zones in Sharded Clusters}
Zones can help improve the locality of data for sharded clusters that span multiple data centers.
\begin{itemize}
    \item In sharded clusters, you can create zones of sharded data based on the shard key.
    \item You can associate each zone with one or more shards in the cluster.
    \item Each zone covers one or more ranges of shard key values.
    \item You must use fields contained in the shard key when defining a new range for a zone to cover.
\end{itemize}

\subsection{Transactions}
Starting in MongoDB 4.2, with the introduction of distributed transactions, multi-document transactions are available on sharded clusters.
\begin{itemize}
    \item Until a transaction commits, the data changes made in the transaction are not visible outside the transaction.
    \item However, when a transaction writes to multiple shards, not all outside read operations need to wait for the result of the committed transaction to be visible across the shards.
\end{itemize}