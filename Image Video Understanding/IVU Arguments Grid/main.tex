\documentclass[12pt,oneside]{report}
\usepackage[utf8]{inputenc}
\usepackage{graphicx}
\usepackage{caption}
\usepackage{subcaption}
\usepackage[a4paper,width=150mm,top=25mm,bottom=25mm,bindingoffset=6mm]{geometry}
\usepackage{fancyhdr}
\usepackage[english]{babel}
\pagestyle{fancy}
\usepackage{biblatex}
\addbibresource{references.bib}
\usepackage{comment}
\usepackage{amsfonts}
\usepackage{tcolorbox}
\usepackage{amsmath}
\usepackage{wrapfig}
\usepackage{tabularx}
\usepackage{csquotes}
\usepackage{adjustbox}
\usepackage{graphicx}
\usepackage{makecell}
\usepackage{footnote}
\usepackage{lscape}
\usepackage{graphicx}
\graphicspath{{figures/}}
\usepackage{array}
\usepackage{threeparttable}
\usepackage{rotating}
\usepackage[titletoc]{appendix}
\DeclareMathOperator*{\argmax}{arg\,max}
\DeclareMathOperator*{\argmin}{arg\,min}
%\usepackage{afterpage}


\newcommand{\newblanckpage}{
    \newpage
    \thispagestyle{plain} % empty
    \mbox{}
}

\begin{document}
\newgeometry{margin=1in}
\begin{titlepage}
    \vspace*{1 cm}
    \begin{center}
         {\LARGE \textbf{Raw Afternotes - Deep Learning for NLP}\\}
        \vspace{2 cm}
        \textsc{Ca' Foscari University of Venice}\\
        Department of Environmental Sciences, Informatics and Statistics\\
        \vspace{0.2 cm}
        \begin{figure}[h!]
        	\centering
        	\includegraphics[width=0.6\textwidth]{logo} 
        \end{figure}
        \vspace{0.0 cm} DEEP LEARNING FOR NATURAL LANGUAGE PROCESSING [CM0624]\\
        %\vspace{0.0 cm} Thesis notes\\
        Academic Year 2023 - 2024\\
        \vspace{3.0 cm}
        	
        \begin{flushleft}
        	\begin{tabular}{l l}
        		\textbf{Student} & Zuliani Riccardo 875532\\
                \textbf{Notes} & These notes do not provide any images, they were made to have a logical ladder
        	\end{tabular}
        \end{flushleft}
    \end{center}
\end{titlepage}

\restoregeometry

%\newblanckpage

\pagenumbering{roman}

%\newblanckpage

\tableofcontents

\clearpage

\pagenumbering{arabic}

\setcounter{page}{1}





\chapter{Introduction}
\section{Theories of vision: from Pythagoras to Marr}
\begin{itemize}
    \item The emission theory
    \item Foundamentally misunderstanding visual perception
    \item The intromission theory
    \item Plato's view
    \item Alhazen's synthesis: book of optics
    \item Kepler's modern theory of retinal images
    \item Nativism vs Empiricism
    \item Helmholtz: vision as unconcius inference
    \item The Gestal School
    \item David MArr ad the computational approach
    \item Missing component
\end{itemize}

\section{Computer Vision}
\begin{itemize}
    \item What is computer vision
    \item Image processing
    \item Pattern recognition
    \item Scene analysis
    \item Block worlds
    \begin{itemize}
        \item Input image
        \item Primal sketch
        \item 2 \textonehalf D sketch
        \item 3D model representation
    \end{itemize}
    \item Edge detection
    \item Image segmentation and clustering
    \item Face Detection: Viola \& Jones
    \item Sift Features
    \item Pascal Visual object challenge
    \item ImageNet
    \item Deep learning philosophy
    \item Inspiration from biology
    \item Foveal vision
\end{itemize}


\chapter{Machine Learning Basics}
\begin{itemize}
    \item What is Machine Learning
    \begin{itemize}
        \item Given n obj, nxn matrix of pairwise similarity
        \item Goal
        \item Usual assumption
    \end{itemize}
    \item Clustering 
    \item Image segmentation as cluster
    \begin{itemize}
        \item Partition the image in choerence region
        \item Group pixels togheter
        \item How compare pixels? measure distance in 3d using euclidean distance
    \end{itemize}
    \item K-Means
    \begin{itemize}
        \item Iterative algorithm, obj represented as feature vector
        \item Algorithm
        \begin{enumerate}
            \item Initialize: pick k random points as cluster center
            \item Alternate: assign data point to closest cluster center, change cluster center to average of its points
            \item Stop: when no changes
        \end{enumerate}
        \item Proprieties:
        \begin{itemize}
            \item Converge
            \item Minimize an objective function, thus convergence is proved
            \item Pros: very efficient, simple
            \item Cons:
            \begin{enumerate}
                \item Converges to local minimum
                \item Need to pick K
                \item Sensitive to initialization
                \item Sensitive to outliers
            \end{enumerate}
        \end{itemize}
    \end{itemize}
\end{itemize}

\section{Supervised Learning a.k.a. Classification}
\begin{itemize}
    \item Classification Problem
    \item Geometric Interpretation
    \item The formal Setup: statistical learning theory deal with supervised learning problems
    \begin{itemize}
        \item Given Input feature space \(X\)
        \item Output label space \(Y = \{-1, +1\}\)
    \end{itemize}
    \item Estimate a functional relation between the input and the output
    \item Training data
    \item Given the training set find a good classifier
    \item Assumption
    \item Loss \& Risk
    \item Bayes Classifier
    \item Bayes' Theorem
    \item The classification problem revisited
    \item The Nearest Neighbour (NN) Rule
    \begin{itemize}
        \item Definition
        \item How good is the NN rule
        \item Variations: K-NN, \(K_n\)-NN
        \item Stone theorem
    \end{itemize}
\end{itemize}

\section{Neural Network}
\begin{itemize}
    \item McCulloch and Pitts Model
    \item Network Topologies and Architectures
    \begin{itemize}
        \item Feedforward only: fully connected and single layer
        \item Feedback networks: sparsely connected and multilayer
    \end{itemize}
    \item Classification Problem
    \begin{itemize}
        \item Features, classes
        \item Finding the best configuration of weights on the income connection and the threshold
        \item Forget the threshold by adding an extra unit set to -1
    \end{itemize}
    \item \textbf{Perceptron:} definition
    \item The perceptron Convergence Theorem
    \item Multilayer feedforward networks
    \begin{itemize}
        \item Single layer
        \item Multilayer feedforward network by adding a hidden layer
        \item \textbf{Universal Approximation Power}
    \end{itemize}
    \item Continuous - values units:
    \begin{itemize}
        \item Sigmoid
        \item Hyperblioc Tangent
    \end{itemize}
    \item \textbf{Back - propagation Learning Algorithm}
    \begin{itemize}
        \item Definition, Supervised Learning
        \item In what consist the learning
        \item Error Function
        \item What is our aim
        \item What do we use to achieve this
        \item \textbf{Pass:}
        \begin{itemize}
            \item Feedforward Pass
            \item Backword Pass
        \end{itemize}
        \item Locality of Back - propagation
        \begin{itemize}
            \item Off - Line
            \item On - Line
            \item Compromise
        \end{itemize}
    \end{itemize}
    \item \textbf{The Algorithm}
    \item Problem of the choice of the learning rate:
    \begin{itemize}
        \item Small
        \item Big
        \item Solution: momentum term: definition and characteristics
    \end{itemize}
    \item Problem of local minima
    \item Theoretical / Practical questions
    \begin{itemize}
        \item Generalization
        \item Training, Validation and Test set
        \item Learning phase stopped in the minimum validation error
    \end{itemize}
    \item Cross validation
    \item Overfitting
\end{itemize}


\chapter{Detecting Faces}
\begin{itemize}
    \item Difficulties
    \item Related Problems:
    \begin{itemize}
        \item Face localization
        \item Face feature extraction
        \item Face recognition
        \item Face Authentication
        \item Face Tracking
        \item Emotion Recognition
    \end{itemize}
    \item Detection: concerned with a category of object
    \item Recognition: concernes with the identity of the object
    \item Methods to detect faces:
    \begin{itemize}
        \item Knowledge-Based Methods: encode human knowledge of what constitutes a typical face
        \item Feature Invariant Approaches: aim to find structural features, invariant in some scenario
        \item Template Matching Methods
        \item Appearence-Based Methods: the models are learned from a set of training images
    \end{itemize}
\end{itemize}

\section{Knowledge-Based Method}
\begin{itemize}
    \item Top down approaches
    \item Multi resolution focus of attention approach
    \begin{itemize}
        \item Level 1: lowest resolution, make the hypothesis
        \item Level 2: local histogram eqalization followed by edge detection
        \item Level 3: search fro eye and mouth for feature validation
    \end{itemize}
    \item Takes horizontal and vertical projection of rows and columns of our image to get an histogram of the pixel level of roes and columns
    \item Pros:
    \begin{itemize}
        \item Easy, simple rules
        \item Facial features in an input image are extracted first
        \item Work well for face localization in uncluttered background
    \end{itemize}
    \item Cons:
    \begin{itemize}
        \item Difficult to translate human knowledge
        \item Different poses of faces are difficut to detect
    \end{itemize}
\end{itemize}

\section{Feature-Based Methods}
\begin{itemize}
    \item Bottom Up approach
    \item Random graph matching
    \item Pro: features are invariant to pose and orientation
    \item Cons:
    \begin{itemize}
        \item Difficult to locate facial features due to several corruption
        \item Difficult to detect features in complex background
    \end{itemize}
\end{itemize}


\section{Template Matching Methods}
\begin{itemize}
    \item Store a template, predefined or deformable
    \item Templates are hard coded
    \item Use correlation to locate faces
    \item Abstraction of the face in term of line drawing, move the template around the image trying to find the best match
    \item Pro: simple
    \item Cons:
    \begin{itemize}
        \item Template must be first initialized near a face image
        \item Difficult to enumerate templates for different poses
    \end{itemize}
\end{itemize}

\section{Appearence-Based Methods}
\begin{itemize}
    \item Collect a large set of face and non face images and train a classifier to discriminate them
    \item given a test image, detect faces by applying the classifier at each position and scale of the image
\end{itemize}

\subsection{Sun and Poggio}
\begin{itemize}
    \item The system was capable of:
    \begin{itemize}
        \item Be invariant on the position of the face (19 x 19 pixels sliding window)
        \item Detect faces in different scale
    \end{itemize}
    \item Preprocessing
    \begin{itemize}
        \item Resizing: 19 x 19
        \item Masking: cropping image angle of the window
        \item Illumination gradient correction: find the best brightness plane and the subtract
        \item Histogram equalization: compensates the imaging effects
    \end{itemize}
    \item Distribution of the face patterns
    \begin{itemize}
        \item Record are grouped in 6 clusters using k-means with k = 6
        \item cluster modified via a multi-dimensional gaussian with cetroid and a covariance matrix
        \item Spectro analysis using the largest eigenvector
    \end{itemize}
    \item Distance Metrics: distances of a sample to all the dace and non face clusters, each distance has two metrics
    \begin{itemize}
        \item Distance between the point and the subspace corresponding to the cluster
        \item Given the projection of the point on the subspace what is the distance between the point and the centroid of the clusters
        \item each face non face is represented by a vector of these two distances
    \end{itemize}
    \item The classifier
    \begin{itemize}
        \item Multilayer NN to identify "face" window patterns from "non face" patterns on their "difference" feature vector of the 12 distance measurements
    \end{itemize}
    \item Create virtual positive examples, data argumentation
    \item Bootstrapping: how can i create a dataset that contains all the non faces -> Algorithm
\end{itemize}

\subsection{Rowley - Baluja - Kandle}
\begin{itemize}
    \item Features:
    \begin{itemize}
        \item Similar to sun and poggio
        \item 20 x 20 instead of 19 x 19
        \item same technique for bootstrapping and pre-processing
        \item NN applied directly to the image
        \item Different heuristic
        \item Faster than sun and poggio
    \end{itemize}
    \item The architecture:
    \begin{itemize}
        \item Neurons as feature detector, each see only a small part of the image and respond in a specific scenario
    \end{itemize}
    \item Problem \& Solution: only detect faces in upper right position
    \begin{itemize}
        \item Image in input to the first NN -> output the probable orientation
        \item Roteate the image with the angle obtained and feed this input to the previous NN
    \end{itemize}
    \item Router Network:
    \begin{itemize}
        \item Roteate a dace sample at 10 degree increment
        \item Create virtual examples from each sample
        \item Train a multilayer NN 
    \end{itemize}
\end{itemize}

\section{The Viola - Jones Face Detector}
\begin{itemize}
    \item Previous algorithm quite slow, create a fast and accurate one
    \item Has the following main 
    \begin{itemize}
        \item Integral images for fast future evaluation
        \item Boosting for feature selection
        \item Attention cascade for fast rejection of non-face windows
    \end{itemize}
    \item Rectangular image features: by nose, mouth, face approx in rect features, dark and lighter
    \item Fast computation with integral images
    \begin{itemize}
        \item sum of pixels value to the top and left of (x,y)
        \item done in 1 pass in linear time starting from the top left corner
        \item cum sum row
        \item integral image
    \end{itemize}
    \item Computing sum within a rectangle
    \begin{itemize}
        \item A,B,C,D
        \item A-B-C+D
    \end{itemize}
    \item Feature selection aka boosting
    \begin{itemize}
        \item 24 x 24 detection region, how learn the best rectangle
        \item boosting build ensamble sequntially
        \item each mdoel corrects the mistakes of its predecessors
    \end{itemize}
    \item Boosting:
    \begin{itemize}
        \item Combination of weak learners
        \item Training consist in multiple boosting round
        \begin{enumerate}
            \item find weak learner with lowest weighted error
            \item raise the weight of missclassifierd examples
        \end{enumerate}
        \item Final classifier is the combination of the weak learner
    \end{itemize}
    \item Boosting for face detection
    \begin{itemize}
        \item Algorithm intuition
        \item Algorithm definition
    \end{itemize}
    \item Attention cascade
    \begin{itemize}
        \item Simple classifier which reject many of the negative sub windows while detecting almost all positive sub-windows
        \item negative outcome at any point leads to the immediate rejection of the sub-window
    \end{itemize}
    \item Non-Maximum suppression
    \begin{itemize}
        \item Set of detections \(\rightarrow\) partitioned \(\rightarrow\) disjoint subset
        \item Two detection are in the same subset if their regions overlap
        \item Each partition yields a single final detection
        \item Corner of the final region are the average of the corners of all detections in the set
    \end{itemize}
\end{itemize}


\chapter{Human Detection}
\begin{itemize}
    \item Challenges
    \item Research issues
    \item The detection phase, sliding window detectors find object in 4 steps
    \item Search over space and scale
    \begin{itemize}
        \item Space: windows is 128 px tall and 64 px wide, 2:1 aspect ration (person viewed from the front and side)
        \item Scale: down-scale the image and slide the window again
    \end{itemize}
\end{itemize}

\section{Support Vector Machines}
\begin{itemize}
    \item Definition
    \item Issue and question
    \item Hyperplane, support vectors, margin
    \item Only support vector thrown away the other examples
    \item Goals:
    \begin{itemize}
        \item 100\% accuracy
        \item maximize the distance respect to the margin
    \end{itemize}
    \item Hyperplane \(w^tx + b = 0\), scalar product does not change it
    \item Canonical from
    \item Optimization problem
    \item Convex quadratic problem, unique minimum
    \item Assumption: linearly separable problem
    \item Slack variable
    \item Regularization term
\end{itemize}


\section{Histogram of Oriented Gradients}
\begin{itemize}
    \item Definition
    \item Steps
    \item Compute Gradients
    \begin{itemize}
        \item Images as continuous function
        \item Derivaty definition and approximation
        \item Gradient vector, magnitude and direction
        \item Horizontal and vertical derivaty by taking the difference of the next and previous points
        \item Overlapping the mask and computing gradients we obtain the edge detection features
    \end{itemize}
    \item Hog Step
    \begin{enumerate}
        \item Horizontal and vector gradients , no pre-processing
        \item Gradient orientation and magnitudes
        \item 64x128 \(\rightarrow\) 16x16 blocks of 50\% overlap
        \item Each block is 2x2 cells with size 8x8
        \item Quantize the gradient orientation into 9 bins
        \item Concatenate histograms
    \end{enumerate}
    \item Classification
\end{itemize}

\section{The Hog Detector-Post Processing}
\begin{itemize}
    \item Non-Maxima Suppression
    \item Are we done? Objects articulated
    \item Two component bycicle model-latent SVMs
\end{itemize}


\chapter{Deep Convolutional Neural Networks}
\begin{itemize}
    \item Learn feature hierarchy from the initial pixel in order to obtain a classifier
    \item Inspiration from biology, receptive fields
    \item Image classification, image, height x weight x 3 of 0 x 255 dimension
    \item Challenge
    \begin{itemize}
        \item Viewpoint 
        \item Illumination 
        \item Scale
        \item Deformation
        \item Background clutter
        \item Occlusion
        \item Intraclass 
    \end{itemize}
    \item Data driven approch
     \item Convolution
    \item Matrix dot product with filter
    \item \textbf{Convolution}
    \item Mask: identity, edge detection, sharpeon, box blur, gaussian blur
    \item Strade, Padding
    \item Traditional approach and deep learning
    \item \textbf{Convolutional Neural Networks (CNN)}
    \begin{itemize}
        \item Object Character Recognition
        \item Fully Connected NN
        \item Locally Connected NN
    \end{itemize}
    \item Maxpooling
    \item \textbf{AlexNET}
    \begin{itemize}
        \item 8 layer in the following schema:
        \begin{enumerate}
            \item Conv + Pool
            \item Conv + Pool
            \item Conv
            \item Conv
            \item Conv + Pool
            \item Full
            \item Full
            \item SoftMax Output (1000 way)
        \end{enumerate}
        \item 2 independent GPU, run in parallel and there are connection between the two GPUs
        \item Way GPU
        \item Deepening in softmax
        \item Using the sigmoid activation function to propagate G, becomes zero
        \item ReLU
    \end{itemize}
    \item Mini-Batch stochastic gradient descent
    \item Technique to reduce overfitting
    \begin{itemize}
        \item Data Argumentation
        \item Dropout
    \end{itemize}
    \item \textbf{ImageNET}
    \begin{itemize}
        \item What is and how it has been build
        \item Overall
    \end{itemize}
    \item Hierarchy of features
    \item Feature analysis
    \item Other Computer vision task
    \begin{itemize}
        \item Semantic segmentation
        \item Classification + Localization
        \item Object Detection
        \item Instance Segmentation
        \item Image Captioning
    \end{itemize}
\end{itemize}

\section{Semantic Segmentation}
\begin{itemize}
    \item Problems
    \item \textbf{First ides: sliding window}
    \begin{itemize}
        \item Classify the central pixel of the patch
        \item Process does not consider contextual information
    \end{itemize}
    \item \textbf{Second Idea: Fully convolutional}
    \begin{itemize}
        \item Feature extractor without pooling layers
        \item Very expensive
    \end{itemize}
    \item \textbf{Third idea: Encoder \& Decoder}
    \begin{itemize}
        \item Encoder: downsapmling
        \item Decoder: upsampling
    \end{itemize}
    \item \textbf{Upsampling}
    \begin{itemize}
        \item Stride and padding as before
        \item Manage overlapping area: sum the overlapping results
        \item Solve "what" but not "where"
        \item Branches
        \item U-Net
    \end{itemize}
\end{itemize}

\section{2D Object detection}
\begin{itemize}
    \item Classification + Localization (ass: single object)
    \begin{itemize}
        \item Classification, soft max
        \item Regression, L2 loss
        \item Losses sum, back propagation
    \end{itemize}
    \item Aside: human pose estimation (ass: single object)
    \begin{itemize}
        \item I want the position of the relevant part of the object
        \item Many output layers, many losses, losses sum, back propagation
    \end{itemize}
    \item Object detection as regression
    \begin{itemize}
        \item Can we relax the assumption of a single predominant object
        \item We have to know in advance the number of object
        \item Not possible
    \end{itemize}
    \item Object detection as classification: sliding window
    \begin{itemize}
        \item Apply the cnn in to huge number of locations
        \item Expensive
    \end{itemize}
    \item R-CNN / Region proposal -selective search
    \begin{itemize}
        \item Find region that are likely to contain objects, selective search
        \item Image
        \item Ad Hoc Training Objective
        \begin{itemize}
            \item Log loss
            \item Hinge loss
            \item Least squares
        \end{itemize}
        \item Training slow, heavy in disk
        \item Inference is slow
    \end{itemize}
    \item \textbf{Fast R-CNN}
    \begin{itemize}
        \item Input the entire image
        \item Process the whole image with several conv and max pool layer, get conv feature map
        \item For each object proposal region of interest pooling layer extract featre vector
        \item FC Layers, two branches:
        \begin{itemize}
            \item Linear + softmax: estimats all k object + "catch-all background class"
            \item Linear: for each of the k object output 4 real number (bounding box regression)
        \end{itemize}
        \item Losses combined by sum and back propagation
    \end{itemize}
    \item \textbf{Faster-RCNN}
    \begin{itemize}
        \item Learn proposal region and is able to recognize them
        \item Insert region proposal network to predict proposal from features
        \item 4 losses from the feature map:
        \begin{itemize}
            \item RPN object / non-object
            \item PRN box coordinates
            \item Final classification score (obj class)
            \item Final box coordinates
        \end{itemize}
    \end{itemize}
    \item \textbf{Recurrent Neural Network}
    \begin{itemize}
        \item Image captioning solved by RNN
        \item Definition
        \item Types
    \end{itemize}
\end{itemize}


\chapter{Graph-Based Methods in Computer Vision: Recent Advances}
\begin{itemize}
    \item Images as graphs
    \item Connection between graphs and matricies
    \item Matrix representation is sensitive to way we number nodes
    \item Eigenvalue and eigenvectors are useful
    \item Invariant respect the way we number the nodes
    \item Edge weights = similarity
    \item Pixel with feature vector \(x\), define distance function for this feature rep
    \item Convert distance between pixels in affinity using gaussial kernel
    \item Formula
    \item \textbf{Clustering on graphs}
    \begin{itemize}
        \item n objects, nxn pairwise similarity matrix A
        \item Goal
        \item Cluster:
        \begin{itemize}
            \item Internal criteria
            \item External criteria
            \item How do we formalize these two
        \end{itemize}
        \item S in or equal to C and i in S
        \item Average weight degree of i with regard the set S
        \item Relative similarity bet i and j respect to the average similarity bet i and its neighbours
        \item Weigh of i with regard to S, gives the similarity bet i and S - i with respect to the overall similarity among the vertices of S - i
        \item The total weight of S 
        \item Definition of Dominant Set
        \begin{itemize}
            \item Internal homogeneity: all node in the cluster are important for it
            \item External homogeneity: considering a new point to add the cluster cohesiveness will decrease
        \end{itemize}
        \item \textit{From dominant set to local optima}
        \begin{itemize}
            \item Cluster as a vector expressing the partecipation of each node to a cluster
            \item \(f(x) = x^TAx\)
            \item Finding x that maximizes f
            \item Normalization, standard simplex
            \item Standard quadratic problem
            \item \(x_i = 0\) or \(x_i >> 0\)
            \item Charateristics
        \end{itemize}
        \item \textit{Using binary symmetric affinities}
        \begin{itemize}
            \item Clique
            \item Maxclique
        \end{itemize}
        \item \textit{Finding dominant sets}
        \begin{itemize}
            \item Replicator dynamics
            \item Formula
        \end{itemize}
        \item Results in image segmentation
        \begin{itemize}
            \item Need more sophisticated features
            \item Similarity measre that takes into account the context of the sorrounding pixels
        \end{itemize}
        \item Replicator dynamics useful for ranking elements in the cluster
        \item Summary of dominant set
    \end{itemize}
    \item \textbf{Detecting conversetional groups in image and sequences}
    \begin{itemize}
        \item F-Formations
        \item Similarity: frustrum of visual attention, definition
    \end{itemize}
    \item \textbf{Dominant set for constrained image segmentation}
    \begin{itemize}
        \item Extract dominant set containing a particular node
        \item Formula
        \item \(\hat{I}_s\) special diagonal matrix
        \item \(\alpha > \lambda_{max}(A_{V/S})\)
        \item Modalities:
        \begin{itemize}
            \item Scribble
            \item Bounding box
        \end{itemize}
        \item Pelillo algo take into account both
        \item Framework:
        \begin{itemize}
            \item Superpixel
            \item Affinity matrix
            \item Replicator Dynamic
            \item Get non zero element from the resulting vector
        \end{itemize}
        \item Bounding box \(\rightarrow\) detect background instead of the foreground
    \end{itemize}
    \item \textbf{Large-Scale image geo-localization using dominant set}
    \begin{itemize}
        \item Problem
        \item Strategy
        \item Pipeline
        \begin{itemize}
            \item Sift features
            \item NN for each feature
            \item NN pruning
            \item DS using global info
            \item CDS
            \item Extract the best matching
            \item Get GPS coordinates
        \end{itemize}
    \end{itemize}
    \item \textbf{Multi-Target Tracking in Multiple Non-Overlapping Cameras using Fast-Constrained Dominant Set}
    \begin{itemize}
        \item Recognize an individual over different non-overlapping cameras
        \item Gallery of person image, recognize a new observed image, probe
        \item \textit{Video-Based Person Re-Identification}
        \begin{itemize}
            \item Traditional Methods
            \item Pelillo's approach
            \begin{itemize}
                \item Bounding box n consecutive frames
                \item Similarity: person appearance and person motion
                \item Graph as many nodes as elements plus the probe
                \item Run CDS to have the probe
                \item Return a rank, take the top one
            \end{itemize}
        \end{itemize}
        \item \textit{Multi-Target Multi-Camera Tracking}
        \begin{itemize}
            \item Track people in each camera
            \item Short tracklets using amount of overlap of bounding box in cons fames
            \item DS same simialrity, return bigger tracklets
            \item DS nodes are the tracklets, return bigger tracklets
            \item Combine all the tracklets, CDS to the camera
        \end{itemize}
    \end{itemize}
\end{itemize}

\section{Context-Aware of Classification}
\begin{itemize}
    \item Context
    \item The Consistent Labelling Problem
    \begin{itemize}
        \item Involves
        \item Goal
        \item Local measurements and contextual information
        \item R
    \end{itemize}
    \item Relaxation Labelling Process
    \item Hummel and Zucker's Consistency
    \item Relaxation Labeling as a non-cooperatibve game
    \begin{itemize}
        \item players = objects
        \item pure strategies = labels
        \item mixed strategies = weighted labeling assignments
        \item payoffs = compatibility coefficients
    \end{itemize}
    \item Graph Transduction
    \begin{itemize}
        \item Description
        \item Goal
        \item Cluster assumption
    \end{itemize}
    \item Graph Transduction Game
    \item Word Sense Disambiguation
    \begin{itemize}
        \item Intended meaning of a word based on the context
        \item Game teoretic model
        \begin{itemize}
            \item players = words
            \item pure strategies = senses
            \item mixed strategies = sense similarity
            \item payoffs = weighted graph
        \end{itemize}
    \end{itemize}
    \item The Protein Function Protection game
    \begin{itemize}
        \item Motivation
        \item Hume's principle
        \item Game:
        \begin{itemize}
            \item players = proteins
            \item strategies = functional classes
            \item payoff function = combination of protein and function-level similarities
        \end{itemize}
    \end{itemize}
    \item \textbf{Metric Learning: Triplet Loss}
    \begin{itemize}
        \item Problem
        \item Triplet Loss
        \item Formula
        \item Triplet Loss Pipeline
        \begin{enumerate}
            \item Prepare data: mini batch of k images
            \item Extract Feature Embedding: DNN
            \item Select Triplets: select subset via its selection method
            \item Evaluate loss using the selected one
        \end{enumerate}
    \end{itemize}
    \item \textbf{Beyond Triplets: The "Group" Loss}
    \begin{itemize}
        \item Initialization: x, softmax, nxn pairwise similarity, nn embedding
        \item Refiniment: refine x, similarity between all mini-batch and labeling preferences
        \item cross-entropy loss, update the weights via backprop
        \item Goal
    \end{itemize}
    \item \textbf{Puzzle Solving with RLP}
\end{itemize}





\end{document}
